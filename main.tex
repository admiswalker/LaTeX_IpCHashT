\documentclass[oneside,openright,uplatex]{jsbook} % oneside: 章ごとに改ページ.twoside: 章の初めが右側となるように改ページ.

% jsbookで余白が広すぎるのを直す
% 参照 https://oku.edu.mie-u.ac.jp/~okumura/jsclasses/
\setlength{\textwidth}{\fullwidth}
\setlength{\evensidemargin}{\oddsidemargin}

% OTF フォントを使えるようにし、複数のウェイトも使用可能にする。
% これがないと、Mac のヒラギノ環境で使われる角ゴが太すぎてみっともない。
\usepackage[deluxe]{otf}

% OT1→T1に変更し、ウムラウトなどを PDF 出力で合成文字ではなくす
\usepackage[T1]{fontenc}

% uplatex の場合に必要な処理 
\usepackage[utf8]{inputenc} % エンコーディングが UTF8 であることを明示する。
\usepackage[prefernoncjk]{pxcjkcat} % アクセントつきラテン文字を欧文扱いにする

% Helvetica と Times を sf と rm のそれぞれで使う。
% default だとバランスが悪いので、日本語に合わせて文字の大きさを調整する。
\usepackage[scaled=1.05,helvratio=0.95]{newtxtext}

\usepackage[dvipdfmx,hiresbb]{graphicx} % 画像の添付に必要
\usepackage[dvipdfmx]{color}    % 画像の添付に必要

% 表でセルを複数列で結合する
\usepackage{multicol}

% 数式の機能を拡張
\usepackage{amsmath}

% citep や citet を有効にする
\usepackage{natbib} % 参考文献を bib で管理するために必要.
\usepackage{url}    % 参考文献のリンク指定オプションに必要.

%\usepackage[hypertex]{hyperref} % url 埋め込みに必要.Usage: \href{url}{text}

% (Okumura, 2009) などを (Okumura 2009) とする
\setcitestyle{aysep={}}

% subfigure 環境で、(a)、(b) などの番号を左上に表示する。宇宙系の分野ではこれが一般的なはず。
\usepackage[nooneline]{subfigure}
\subfiguretopcaptrue

% 行番号を表示する。添削時のみに使い、事務提出版ではコメントアウトする
%\usepackage{lineno}
%\linenumbers

% PDF 内で外部リンクや文書内リンクを生成したい場合に使う(好みによる)
% \usepackage[dvipdfmx]{hyperref}

% newcommand を使うことで、繰り返し使う長ったらしい入力を簡単にすることができる
\makeatletter
\newcommand{\ion}[2]{#1$\;${\small\rmfamily\@Roman{#2}}\relax}%
\makeatother
\newcommand{\HI}{\mbox{\ion{H}{1}}} % 中性原子ガス(HI 領域)の例
\newcommand{\bs}{\symbol{92}} % backslash


%%%%%%%%%%%%%%%%%%%%%%%%%%%%%%%%%%%%%%%%%%%%%%%%%%%%%%%%%%%%%%%%%%%%%%%%%%%%%%%%%%%%%%%%%%%%%%%%%%%%%%%%%%%%%%%%%%%%%%%%%%%%%%

% 図を通し番号にする: http://rexpit.blog29.fc2.com/blog-entry-99.html
\usepackage{remreset}	% removefromreset に使う
\makeatletter
	\@removefromreset{figure}{chapter}
	\def\thefigure{\arabic{figure}}

	\@removefromreset{table}{chapter}
	\def\thetable{\arabic{table}}

	\@removefromreset{equation}{chapter}
	\def\theequation{\arabic{equation}}
\makeatother

\usepackage{comment} % \begin{comment} \end{comment}
\usepackage{color}   % {\bf \color{red}メモ}

\usepackage{natbib} % citep や citet を有効にする
\bibpunct[:]{[}{]}{,}{a}{}{,} % 括弧を () -> [] へ変更
%\usepackage[numbers]{natbib} % citep や citet を有効にする
%\bibpunct[:]{(}{)}{,}{a}{}{,}

\usepackage[]{multicol}         % 段組み % \begin{multicols}{2}%2段組み開始 ~ \end{multicols}%2段組み終了

\usepackage{wrapfig}            % 図の回り込みを許す.\begin{figure}~\end{figure} -> \begin{wrapfigure}{r}{90mm}~\end{wrapfigure} へ置き換え
                                % 直後の文が制御記号だと,回り込みに失敗し,文字が図下に重なるので,制御記号の直前に ''\noindent \ \ '' を挿入して回避する.

%\setlength\floatsep{0pt}         % 図と図の間の余白
%\setlength\textfloatsep{0pt}     % 本文と図の間の余白
%\setlength\intextsep{0pt}        % 本文中の図の余白
\setlength\abovecaptionskip{-3pt} % 図とキャプションの間の余白

%\usepackage{here} % \begin{figure}[H]

\renewcommand\thefootnote{\arabic{footnote})} % footnote % 脚注の調整
%\usepackage{dblfnote} % 脚注を 2 段組にする

%%%%%%%%%%%%%%%%%%%%%%%%%%%%%%%%%%%%%%%%%%%%%%%%%%%%%%%%%%%%%%%%%%%%%%%%%%%%%%%%%%%%%%%%%%%%%%%%%%%%%%%%%%%%%%%%%%%%%%%%%%%%%%

% ソースコードの埋め込み
\usepackage{listings}
\lstset{%
  language={C},
  basicstyle={\small},%
  identifierstyle={\small},%
  commentstyle={\small\itshape},%
  keywordstyle={\small\bfseries},%
  ndkeywordstyle={\small},%
  stringstyle={\small\ttfamily},
  frame={single}, % leftline,topline,bottomline,tb OR lines, trBL,tRBl OR shadowbox, single
  breaklines=true,
  columns=[l]{fullflexible},%
  xrightmargin=0zw,%
  xleftmargin=3zw,%
  %numbers=left,%
  %numberstyle={\scriptsize},%
  %stepnumber=1,
  %numbersep=1zw,%
  tabsize=4, %タブの大きさ
  lineskip=-0.5ex, %行間
  linewidth=16.5cm %フレームの横幅
}

%\usepackage{listings}
%\lstset{
%  basicstyle={\ttfamily\small}, %書体の指定
%  frame=single, % frame=tRBl, %フレームの指定
%  framesep=3pt, %フレームと中身(コード)の間隔
%  breaklines=true, %行が長くなった場合の改行
%  %linewidth=12cm, %フレームの横幅
%  lineskip=-0.5ex, %行間の調整
%  tabsize=4 %Tabを何文字幅にするかの指定
%}

%\usepackage{jlisting}
%\usepackage{listings}
%\usepackage{color}
%\definecolor{OliveGreen}{rgb}{0.0,0.6,0.0}
%\definecolor{Magenta}{cmyk}{0, 1, 0, 0}
%\definecolor{colFunc}{rgb}{1,0.07,0.54}
%\definecolor{CadetBlue}{cmyk}{0.62,0.57,0.23,0}
%\definecolor{Brown}{cmyk}{0,0.81,1,0.60}
%\definecolor{colID}{rgb}{0.63,0.44,0}
%\lstset{
%language={Matlab},                   %言語の指定
%basicstyle={\ttfamily\small},        %書体の指定
%backgroundcolor={\color[gray]{.95}}, %背景色と透過度
%keywordstyle={\color{blue}},         %キーワード(int, ifなど)の書体指定
%commentstyle={\color{OliveGreen}},   %注釈の書体 
%stringstyle=\color{Magenta},         %文字列
%frame=single,                        %枠縁(leftline,topline,bottomline,lines,trBL,shadowbox, single)
%numbers=left,                        %行番号表示
%numberstyle={\ttfamily\small},       %行番号の書体指定
%breaklines=true,                     %折り返し(自動改行)
%breakindent = 10pt,                  %自動改行後のインデント量(デフォルトでは20[pt])	
%tabsize=2,                           %タブの大きさ
%captionpos=t                         %キャプションの場所(t,b : "tb"ならば上下両方に記載)
%}
%\renewcommand{\lstlistingname}{Code} % キャプション名の指定

%%%%%%%%%%%%%%%%%%%%%%%%%%%%%%%%%%%%%%%%%%%%%%%%%%%%%%%%%%%%%%%%%%%%%%%%%%%%%%%%%%%%%%%%%%%%%%%%%%%%%%%%%%%%%%%%%%%%%%%%%%%%%%

\begin{document}

\begin{titlepage}


\vspace*{120truept}
\begin{center}
  \huge{学 \ 士 \ 論 \ 文 / \ 修 \ 士 \ 論 \ 文}\\
  \vspace{30truept}
%\textbf{
  \huge{タイトルが\\長い場合はいい感じに改行}\\ % title
%  \LARGE{------サブタイトル------}\\ % サブタイトル(なければコメントアウト)
%}
\vspace{100truept}
\LARGE{xx大学大学院xxxx研究科}\\
\LARGE{xxxx専攻xxxx分野xx研究室}\\
\LARGE{苗字 名前}\\
\end{center}


\end{titlepage}
 % 表紙
\thispagestyle{empty} % ページ番号を表示しない.

% レイアウトを他の章のはじめのページと揃える.
 \\
\\
\\
\\
\noindent{\Huge \sf 要旨}\\
\\
\\
\\

ハッシュテーブルは,
ハッシュ化された入力キーを配列サイズで除算した際の余り (modulo) を配列のインデックスとして値を格納することで,
定数時間で高速にキーに対応する値を取得できるアルゴリズムである.
ハッシュテーブルは,キーと対応した値を保持する表であるため,配列のことをテーブルと表現する.

一般に,
ハッシュ値の剰余は配列全体に均一に分布することが望ましいが,
実際には限られた配列長に丸めるため衝突する.
%衝突確率を下げるには,より広い値空間に移せばよく,この操作をリハッシュと呼ぶ
%\footnote{リハッシュを行う際は,リハッシュ時間を定数時間に収めるため,通常は倍サイズの配列長に遷移させる.
%ただし,ここでの定数時間とは,繰り返し遷移させた場合のコストを1要素ごとで分割したコストが,単に O(n) となることを意味する.}
%.一般に,Chain 法.
そのため,衝突発生の際は,より広いハッシュ空間に移せばよく,この操作をリハッシュと呼ぶ.
しかし,衝突が偶発的であり,単純にリハッシュを繰り返せば,不用意にメモリを消費することから,
ハッシュ値が衝突した際にも,配列の利用率を示す座席利用率 (load factor) を上げる試みが行われてきた.

ハッシュ値が衝突した際に,key-value ペアを配列に押し込む手法は,大きく2つに分類される.
1つ目は,Chain 法に代表される Closed Addressing (別名:Open Hash) である.


2つ目は, 法に代表される Closed Addressing (別名:Open Hash) である.






\frontmatter     % make page number roman
\tableofcontents % 目次
%\listoffigures  % 図目次
%\listoftables   % 表目次

\mainmatter      % make page number arabic
\chapter{Introduction/序論}
\label{chap_Introduction}

Introduction/序論

\section{Review}
先行研究\footnote{脚注はこのように挿入します.}.

\begin{figure} % 特に強い理由がない限り、[htbp]のような指定はしないでください。
  \centering
%  \includegraphics[width=4.9cm]{./figs/sin.png}
  \includegraphics[width=15cm]{./figs/sin.png}
  \caption{
    Sin 波.区間 [0, 60] に発生している.図の説明は,「名詞句.文章.」の順番に説明するとよい.
    ここでは PNG 画像を貼り付けているが,pdf 等のベクトル図を貼り付けると,
    描画も印刷も格段に綺麗になる.pdf 形式でのグラフプロットは,Matplotlib の出力する拡張子を pdf に置き換えるだけである.
    なお,本画像は一部加工されており,出典は \citep{ADMIS2018} である.
  }
  \label{fig_ADMIS2018}
\end{figure}

\section{Purpose}
研究目的.


\chapter{アルゴリズム}
\label{chap_Algorism}

Linear probing や quadratic probing など
従来の closed hashing では,
ハッシュ先が重複していない場合でも,
別のキーの退避先として配列要素が使用中の場合,
本来 1 回目の探査でアクセスできるキーであっても,
probing により衝突を解決しなければならない.
Robin Hood hashing は,1 回でアクセスできる位置に要素を移動できるものの,
ハッシュ先の重複\footnote{secondary clustering.} に対しては線形探査が必要となる.

本投稿では,
ハッシュ先が別のキーの退避先として利用されてしまう primary clustering に付随した問題に対して,
逆方向にリストを辿ることで,要素が使用済みの場合でも適切に移動し,
ハッシュ先が必ず 1 つ目の要素となるように調整する.
また,要素探査時は,
順方向にリストを辿ることで平均探査回数を削減する.
メモリ使用量を削減するため,リストは int 型により相対位置を記録し,
これら順方向と逆方向を合わせて双方向リスト構造とする.
以上のアルゴリズムを実現するハッシュテーブルを in-place chained hash table と呼ぶこととし,
以降,IpCHashT と略記する.

IpCHashT のデータ構造における提案は,本質的に \cite{ADMIS2017} の提案と同様である.
以下,
Fig. \ref{fig_IpCHashT_struct},
Fig. \ref{fig_IpCHashT_insert_hard_case01}〜\ref{fig_IpCHashT_insert_hard_case11},
Fig. \ref{fig_IpCHashT_deletion_case01}〜\ref{fig_IpCHashT_deletion_case06}は,
\cite{ADMIS2017} に由来する.
ただし,本投稿では,
挿入アルゴリズムに新しく soft insertion を採用している.
探査においても,
successful search を優先するオプションと,
unsuccessful search を優先するオプションについて新たに実装と評価を行う.
また,パディングサイズは従来固定であったが,
テーブルサイズによってパディングサイズを動的に変更することで,
テーブルサイズが小さい場合に最大 load factor が低下する問題に対処する.
要素の挿入においては,操作手順を再検討し,より簡潔な実装とする.

IpCHashT では,Fig. \ref{fig_IpCHashT_struct}に示すデータ構造を備える.
各要素には key-value ペアの他に,双方向リストのための prev 要素と next 要素を持つ.
T\_shift には uint8 または uint16 型を用い,相対位置による双方向リストを構成する.
これは,ポインタ接続におけるメモリ消費量が無視できないためである
\footnote{
  例えば 64 bits CPU の場合,ポインタサイズは 64 bits であるから,T\_key と T\_val が uint64 型の場合,
  テープルサイズの 50\% が双方向リストに由来する.
}.
%\leavevmode \newline

\begin{figure} % 特に強い理由がない限り,[htbp]のような指定はしないでください。
\begin{lstlisting}[language=C++]
template <class T_key, class T_val, typename T_shift>
struct element{
	element(){
		T_shift maxShift = (T_shift) 0; maxShift =~ maxShift;
		prev = maxShift;
		next = (T_shift) 0;
	}
	~element(){}
	
	T_key key;
	T_val val;
	T_shift prev;
	T_shift next;
};
\end{lstlisting}
\caption{
  Pseudo C++ data structure for IpCHashT element. ``T\_key'' is a key type and ``T\_val'' is val a value type.
  ``T\_shift'' is used for doubly linked lists and specifies uint8 or uint16.
  Specifying a type larger than uint16 has little merit.
  ``prev'' indicates the distance from the previous element and 
  ``next'' indicates the distance from next element.
  Unlike the address in the pointer, the link list is expressed in the interval $[0, {\rm max(T\_shift)}-1]$,
  then max(T\_shift) is the maximum possible value of T\_shift type.
  ``$0$'' indicates itself and ``${\rm max(T\_shift)}-1$'' is the maximum link distance.
  There is no method to link outside of a section.
  ``max(T\_shift)'' is reserved for prev, and ``prev==max(T\_shift)'' indicates that the element is empty.
  ``prev==0'' indicates that the element is the head of the linked list,
  and ``next==0'' indicates that the element is the tail of the linked list.
  \\
  %%%
  {\bf 図\ref{fig_IpCHashT_struct}}
  IpCHashT 要素の C++ 擬似構造体.T\_key はキーの型,T\_val は val の型である.
  T\_shift は双方向リストに用いる型で,uint8 または uint16 を指定する.
  uint16 より大きな型を指定するメリットは殆どない.
  prev は 前の要素までの相対距離を,next は 次の要素までの相対距離を表す.
  ポインタにおけるアドレスとは異なり,区間 [0, max(T\_shift) - 1] の範囲でリンクを表現する.
  ただし,max(T\_shift) は T\_shift 型の取り得る最大値である.
  0 のときに自分自身を示し,max(T\_shift) - 1 がリンクできる最大距離である.区間外へのリンクはできない.
  max(T\_shift) は,予約されており,'prev==max(T\_shift)' のとき,要素が空であることを示す.
  また,'prev==0' であればリストの先頭であること,'next==0' であればリストの末尾であることがわかる.
}
\label{fig_IpCHashT_struct}
\end{figure}

\begin{figure}
  \includegraphics[scale=0.73]{./fig_algo/algorism_crop_01.pdf}
  \caption{
    Symbols used in Fig. \ref{fig_IpCHashT_apparence}〜\ref{fig_IpCHashT_deletion_case06}.
    `` Ope. '' indecates the execution order, and `` Ope. 0 '' means the initial state.
    The down arrow is placed on the array element indicated by the hash destination.
    Each element is composed by a circle, a prev locator, and a next locator.
    However, locators are omitted when there are no connection.
    Circles written with dotted lines represent empty elements.
    Green boxes and arrows represent movement of elements.
    Crosses indicate that the linke is deleted.
    The color scheme is blue for the initial state, red for element insertion and deletion, and green for element movement.
    \\
    {\bf 図\ref{fig_IpCHashT_fig_description}}
    Fig. \ref{fig_IpCHashT_apparence}〜\ref{fig_IpCHashT_deletion_case06}に用いる記号.
    ``Ope.''は,実行順序を表し,``Ope. 0'' の場合は初期状態を意味する.
    下向き矢印は,ハッシュ先が示す配列要素の上に置かれる.
    各要素は,丸 1 つと prev locator 1 つ,next locator 1 つで構成される.
    ただし,接続が無い場合 locator は省略される.
    点線で書かれた丸は空の要素を表す.また,緑色の枠線と矢印は要素の移動を表す.
    バツ印はリストの削除表す.
    配色は,青色を初期状態,赤色を要素の挿入と削除,緑色を要素の移動,としている.
  }
  \label{fig_IpCHashT_fig_description}
%\end{figure}
%\begin{figure}
  \includegraphics[scale=0.73]{./fig_algo/algorism_crop_03.pdf}
  \caption{
    An abstract representation of the element chain inserted into IpCHashT.
    In this case, the hash destinations of the three elements indicate the address of the first element,
    and the conflict is resolved by doubly linked list.
    Each element is connected by prev and next locators that indicate relative position, 
    and there may be another element between each element.
    \\
    {\bf 図\ref{fig_IpCHashT_apparence}}
    IpCHashT に挿入された要素の抽象表現.
    この場合,3 つの要素のハッシュ先は,いずれも first 要素のアドレスを示すため,
    双方向リストにより衝突を解決している.
    各要素は prev locator と next locator の示す相対位置により接続されており,
    各要素間に別の要素がある可能性がある.
  }
  \label{fig_IpCHashT_apparence}
%\end{figure}
%\begin{figure}
  \includegraphics[scale=0.73]{./fig_algo/algorism_crop_04.pdf}
  \caption{
    An example of mapping the abstract representation shown in Fig. \ref{fig_IpCHashT_apparence} to a continuous address space.
    A gray frame represents the array.
    Even with the same abstract representation,
    the state of the continuous address space depends on the array state at the time of insertion.
    Also, deleting elements can cause memory fragmentation.
    The blue list is a map of the abstract representation shown in Fig. \ref{fig_IpCHashT_apparence},
    and the red list is another chain inserted between them.
    In this example,
    the blue list is fragmented and the tail element is inserted in a distant place
    even though there is an empty element in the foreground.
    This fragmentation occurs when three or more elements are inserted into the blue list
    and then the value stored in the left empty element is deleted.
    \\
    {\bf 図\ref{fig_IpCHashT_apparence}}
    Fig. \ref{fig_IpCHashT_apparence}に示す抽象表現を連続アドレス空間上に写像した一例.
    灰色の枠線は配列を表す.
    同じ抽象表現でも,連続アドレス空間の状態は,要素挿入時の配列状態により異なる.
    また,要素の削除はメモリを断片化させることがある.
    青色のリストはFig. \ref{fig_IpCHashT_apparence}に示す抽象表現の写像であり,
    赤色のリストは間に挿入された別の chain である.
    この例では,青色のリストが断片化されており,手前に空の要素があるにも関わらず,末尾要素が遠い場所に格納されている.
    この断片化は,3 つ以上の要素が青色のリストに挿入された後に,
    左側の空要素に格納されていた値が削除されたことにより発生する.
  }
  \label{fig_IpCHashT_insert_introspection}
\end{figure}

本章で示すFig. \ref{fig_IpCHashT_fig_description}〜\ref{fig_IpCHashT_deletion_case06}は,
青色を初期状態とし,赤色が挿入時の変更を,緑色は要素の移動に伴う変更を,それぞれ表す.
Fig. \ref{fig_IpCHashT_fig_description}では,
Fig. \ref{fig_IpCHashT_apparence}〜\ref{fig_IpCHashT_deletion_case06}に用いる記号を説明している.
Fig. \ref{fig_IpCHashT_apparence}〜\ref{fig_IpCHashT_deletion_case06}では,
丸印ひとつ 1 つが配列要素を,
双方向に張られた2つの矢印が双方向リストを表す.
Open hashing を用いたハッシュテーブルの説明では,
連続したアドレス空間上の配列要素が,
2辺を共有した四角形の連なりにより表現されることが多い.
同様に,本投稿では IpCHashT の配列要素を抽象化して表すが,
一見するとFig. \ref{fig_IpCHashT_apparence}のように隣接した連続構造であっても,
実際には,Fig. \ref{fig_IpCHashT_insert_introspection}のように各要素が断続的なアドレス空間上に存在することがある.

%\leavevmode \newline
%\vspace{-1cm}
\leavevmode \newline
\section{挿入}

高い探査性能を達成するためには,
命令数を削減するだけでなく,
キャッシュミスを抑える必要がある.
一般に,CPU は配列アクセスに対して,
参照の局所性を利用してキャッシングする.
そのため,
必要な要素を配列に隙間なく詰め込むことで空間的局所性を,
可能な限り連続した位置に配置することで逐次的局所性を,
それぞれ高め,キャッシュミスを削減する.

key-value ペアを 1 つ挿入するには,
次の 1) 〜 4) の場合を考える.
1) ハッシュ先の配列が空の場合は,Fig. \ref{fig_IpCHashT_insert_hard_case01}のように単に要素を詰める.
2) 既に要素が挿入されており chain の間に空きがある場合は,Fig. \ref{fig_IpCHashT_insert_hard_case02}のように間に挿入する.
3) chain の間に空きがない場合は,Fig. \ref{fig_IpCHashT_insert_hard_case03}のように末尾に空きを探し挿入する.
4) 挿入先の要素が異なるハッシュ値を持つ要素の退避先に使用されている場合は,
Fig. \ref{fig_IpCHashT_insert_hard_case04}〜\ref{fig_IpCHashT_insert_hard_case11}のように挿入する.
要素の退避先と locator の再接続先との兼ね合いのため,4) には多くの場合分けが必要となる.

chain 間に空きは,線形探査により調べる.
これには,対象の要素を全て調べ上げる必要があり,処理に時間が掛かる.
また,11 通りの場合分けを全て実装するには非常に手間も掛かる.
Fig. \ref{fig_IpCHashT_insert_hard_case01}〜\ref{fig_IpCHashT_insert_hard_case11}に示す場合分けは,
任意の箇所に空きがあることを考慮している.
しかし,要素削除なく挿入する場合には,メモリの断片化は発生しないため,
必要な場合分けは
Fig. \ref{fig_IpCHashT_insert_hard_case01},
\ref{fig_IpCHashT_insert_hard_case03},
\ref{fig_IpCHashT_insert_hard_case06},
\ref{fig_IpCHashT_insert_hard_case11}
の 5 通りである.
また,要素削除を伴う場合においても,挿入コストが高いため推奨しない.
これは,要素の削除と再挿入によるメモリ断片化への耐性を捨て,
実装コストの低減と,挿入の高速化を優先している.


\section{探査}

挿入済みの要素を探査するには,Fig. \ref{alg_find_sm},\ref{alg_find_usm}に示すアルゴリズムが考えられる.
Fig. \ref{alg_find_sm}は successful search を優先した設定であり,以降,successful search major option と呼ぶ.
Fig. \ref{alg_find_usm}は unsuccessful search を優先した設定であり,以降,unsuccessful search major option と呼ぶ.
\samepage\newline\indent
Fig. \ref{alg_find_usm}の unsuccessful search major option は,
キーの比較コストによって successful search major option の結果より悪化することが十分に考えられるが,
比較コストの低いキーの場合には,unsuccessful search 時の分岐が少なく,高い性能が期待される.
実際に,分岐予測の失敗におけるペナルティは,10\textasciitilde 20 clock \footnote
{
  Fig. \ref{fig_part101_infographics}の ``Wrong'' branch of ``if'' (branch misprediction) を参照.
}
程度であり,無視できない.

\begin{figure}%[h] % 特に強い理由がない限り,[htbp]のような指定はしないでください。
\begin{lstlisting}[language=C++]
template <class T_key, class T_val>
(bool, T_val) find(T_key key_in){
	uint64 hashVal = hashFunc( key_in );
	uint64 idx = hashVal & tableSize_minus1;
	
	if(! isHead( table[ idx ] ) ){ return ( false, none ); }
	for(;;){
		if( table[ idx ].key == key_in ){ return ( true, table[ idx ].val ); }
		if( table[ idx ].next == 0 ){ return ( false, none ); }
		idx += table[ idx ].next;
	}
}
\end{lstlisting}
\caption{
  Pseudo C++ code for the ``find()'' function executed while {\bf successful search major option} is specified.
  This function is tuned to speed up ``successful searches'' more than ``unsuccessful seatches''.
  The ``isHead()'' function checks the hash destination is a head element or not, and if it is the head, it follows the list.
  At this time, the ``isHead()'' function returns true
  when the ``prev'' element which is a member element of the structure shown in Fig. \ref{fig_IpCHashT_struct} is $0$.
  Then the ``find()'' function returns true and a corresponding value to the key when a ``key\_in'' is equal to the key on the array.
  In all other cases, the ``find()'' function returns false.
  Compared with the existing closed hashing method,
  the number of key comparisons is limited to the number of elements on the doubly linked list,
  therefore high-speed successful search is possible.
  As with the dense\_hash\_map,
  the check with the ``isHead()'' function becomes unnecessary
  when one key is registered as an empty mark and is guaranteed not to be used.
  This is because the initial value of an element outside the list is guaranteed not to match any key.
  For this reason, removing the ``isHead()'' function will speed up unsuccessful search too while the key comparison cost is small.
  However, this post does not deal with such special conditions.
  \\
  {\bf 図\ref{alg_find_sm}}
  Successful search major option を指定した場合に実行される find 関数の擬似 C++ コード.
  Successful search が高速に実行されるようにチューニングされている.
  isHead 関数によりハッシュ先の要素が双方向リストの先頭か否かを確認し,先頭の場合に双方向リストを辿る.
  このとき,isHead 関数は,Fig. \ref{fig_IpCHashT_struct} に示す element 構造体の prev 要素が 0 のときに true を返す.
  リストの先頭の場合,key\_in が配列上のキーと一致すれば,対応する値と true を返す.
  それ以外の場合は全て false を返す.
  既存の closed hashing 法と比較して,キーの比較回数が双方向リスト上の要素数に制限されるため,高速な successful search が可能となる.
  なお,dense\_hash\_map と同様に,あるキーを空符号として登録し,使用されないことが保証される場合は,isHead 関数による検査が不要となる.
  これは,リスト外の要素の初期値が,いずれのキーとも一致しないことが保証されるためである.
  このため,isHead 関数を削除すると,キーの比較コストが小さい場合に,unsuccessful search も高速になる.
  ただし,条件が特殊なため,この投稿では扱わない.
}
\label{alg_find_sm}
\end{figure}

\begin{figure}%[h] % 特に強い理由がない限り,[htbp]のような指定はしないでください。
\begin{lstlisting}[language=C++]
template <class T_key, class T_val>
(bool, T_val) find(T_key key_in){
	uint64 hashVal = hashFunc( key_in );
	uint64 idx = hashVal & tableSize_minus1;
	
	for(;;){
		if( table[ idx ].key == key_in ){
			if( isEmpty( table[ idx ] ) ){ return ( false, none ); }
			return ( true, table[ idx ].val );
		}
		if( table[ idx ].next == 0 ){ return ( false, none ); }
		idx += table[ idx ].next;
	}
}
\end{lstlisting}
\caption{
  Pseudo C++ code for the ``find()'' function executed while {\bf unsuccessful search major option} is specified.
  This function is tuned to speed up ``unsuccessful searches'' more than ``successful seatches''.
  However, since the number of key comparisons is greater than that of successful search major options,
  data structures with high key comparison costs are considered to perform worse than successful search major options.
  In a case of unsuccessful search, it is necessary to find out all the keys associated with hash destination do not match ``key\_in''.
  Prioritizing key comparison makes unsuccessful search faster than ensuring the element is head.
  However, comparing the keys even if the hash destination is not a head of the list,
  comparing keys will cause a malfunction when a initial value of key element is equal to key\_in.
  For this reason, ``isEmpty()'' needs to check that the element is not empty.
  \\
  {\bf 図\ref{alg_find_usm}}
  Unsuccessful search major option を指定した場合に実行される find 関数の擬似 C++ コード.
  Unsuccessful search が高速に実行されるようにチューニングされている.
  ただし,キーの比較回数は successful search major option よりも多くなるため,
  キーの比較コストが高いデータ構造では,successful search major option より性能が悪化すると考えられる.
  Unsuccessful search では,キーの一致の有無に関わらず,ハッシュ先に関係するキーを全て探査し,key\_in と一致しないことを確認する必要がある.
  そのため,isHead より先頭要素であるかを確かめるよりも,キーの比較を優先する方が,高速に動作する.
  ところが,リストの先頭でない場合もキーを比較するため,
  キーが要素の初期値と等しい場合に誤動作する.
  このため,isEmpty により要素が空でないことを確認する必要がある.
}
\label{alg_find_usm}
\end{figure}

%\begin{algorithm}
%  \caption{Calculate $y = x^n$}
%  \label{alg1}
%  \begin{algorithmic}
%    \REQUIRE $n \geq 0 \vee x \neq 0$
%    \ENSURE $y = x^n$
%    \STATE $y \Leftarrow 1$
%    \IF{$n < 0$}
%    \STATE $X \Leftarrow 1 / x$
%    \STATE $N \Leftarrow -n$
%    \ELSE
%    \STATE $X \Leftarrow x$
%    \STATE $N \Leftarrow n$
%    \ENDIF
%    \WHILE{$N \neq 0$}
%    \IF{$N$ is even}
%    \STATE $X \Leftarrow X \times X$
%    \STATE $N \Leftarrow N / 2$
%    \ELSE[$N$ is odd]
%    \STATE $y \Leftarrow y \times X$
%    \STATE $N \Leftarrow N - 1$
%    \ENDIF
%    \ENDWHILE
%  \end{algorithmic}
%\end{algorithm}


\section{削除}

通常,リストの要素を削除するには,要素を削除した上で,ポインタを繋ぎ変えればよい.
しかし,IpCHashT では,幾つかの場合を考慮する必要がある.

key-value ペアを 1 つ削除するには,
次の 1) 〜 4) と,メモリの断片化を防ぐ処理として 5) を考える.
1) 単一要素の場合は,Fig. \ref{fig_IpCHashT_deletion_case01}のように単に削除する.
2) 末尾要素を削除する場合は,Fig. \ref{fig_IpCHashT_deletion_case02}のように要素と locator を削除する.
3) 先頭の要素が削除された場合,探査不能としないため,先頭に別の要素をつなぎ替える必要がある.
Fig. \ref{fig_IpCHashT_deletion_case03}では,断片化を防ぐために,末尾のデータを先頭へ移動させている.
4) chain の中間要素を削除する場合は,Fig. \ref{fig_IpCHashT_deletion_case04}のように末尾要素を移動させる.
5) 別の削除処理によって,末尾の要素が移動すると,
Fig. \ref{fig_IpCHashT_deletion_case05},\ref{fig_IpCHashT_deletion_case06}のように,
chain の要素間に空きができ,断片化する場合がある.これを
Fig. \ref{fig_IpCHashT_deletion_case05},\ref{fig_IpCHashT_deletion_case06}に示す操作を繰り返すことにより修正する.
ただし,空き要素の探査は線形探査する必要がある.
また,要素を移動させる度に別の chain に空きができる可能性があるため,再帰的に行う必要がある.
この処理は,実行コストが高いため推奨しない.

以上を勘案して,1) 〜 4) の操作を実装する.5) は実装しない.

% \section{断片化}
\section{配列の末尾処理}

要素の衝突が発生した際,closed hashing では,
現在より後のアドレスに key-value ペアを格納することで衝突を解決する.
しかし,ハッシュ値が配列の末尾を示した場合は,
退避先の配列要素が存在しない.
この場合,配列の末尾に達した場合は,
1) 引き続いて先頭から辿るように処理する,
2) 予め末尾に余分な配列を確保する,
の二択である.
まず,1) は,配列の読み込みが不連続となりキャッシュミスを誘発する.このため採用できない.
次に,2) は,キャッシュミスを誘発する危険はないものの,
余分な配列をどの程度確保するか問題となる.

2) で最も簡単な実装は,パディングを固定長とすることである.
ただし,これには欠点があり,テーブルサイズが小さいとき,
パディングが不足すると衝突を解決できずにリハッシュが発生し,
load factor の上限は上がらない,
逆に,パディング過剰の場合,
不用意にリハッシュが抑制され,探査効率が落ちる.

テーブルサイズが小さい場合,
パディングサイズが大きいいと,
テーブルサイズよりもパディングサイズが支配的となる.
逆に,テーブルサイズが大きい場合,
パディングサイズが小さいと,
退避先の配列が不足し,
最大 load factor が悪化する.
したがって,パディングサイズは,
テーブルサイズに応じて調整が必要となる.

必要なパディングサイズは,
線形に増加すると推測されるため,
\[
  {\rm pSize} = \begin{cases}
    (1/a) \cdot {\rm tSize} & ({\rm tSize}<{\rm limit}) \\
    {\rm limit}    & ({\rm tSize}\geq{\rm limit})
  \end{cases}
\]
\[
  a: {\rm threshold}, \ \ 
  {\rm pSize}: {\rm padding \ size}, \ \ 
  {\rm tSize}: {\rm table \ size}, \ \ 
  {\rm limit}: {\rm limit \ of \ T\_shift \ size}
\]
とすればよい.
しかし,
テーブルサイズが小さい場合ハッシュ先は十分に分散しないため,
単純に原点を通した調整では load factor の上限値が安定しない.
したがって,先ほどの数式に,バイアス項を付与した
\[
  {\rm pSize} = \begin{cases}
    (1/a) \cdot {\rm tSize} + b & ({\rm tSize}<{\rm limit}) \\
    {\rm limit}    & ({\rm tSize}\geq{\rm limit})
  \end{cases}
\]
\[
  a: {\rm threshold}, \ \ 
  b: {\rm bias}, \ \ 
  {\rm pSize}: {\rm padding \ size}, \ \ 
  {\rm tSize}: {\rm table \ size}
  {\rm limit}: {\rm limit \ of \ T\_shift \ size}
\]
をパディングの調整に用いる.

このとき,定数 $a$, $b$ は,load factor を最大化する,最小のパディングサイズとなる値が望ましい.
この定数が不用意に大きいと,IpCHashT は T\_shift の限界まで要素を挿入しようとする.
これは,特に,テーブルサイズが小さい場合に顕著となる.
また,uint16 を用いる場合は,その最大 T\_shift サイズ - 1 の 65534 要素先まで接続できてしまうため,
影響が広い区間に渡って続くことになる.
なお,この場合 65535 は空フラグとして予約されている.

定数 $a$, $b$ の最適値はともかくとして,
実用に耐えうる定数を探すだけであれば,
実験的に求められる.
これには,Fig. \ref{fig_bench_LF}に示す IpCHashT<uint64,uint64> (uint8, maxLF100) の load factor が最大となる,
最小のパディングサイズを探せばよい.
例えば,$a=18$, \ $b=35$, \ ${\rm limit}=254$ である.
なお,254 以上のパディングは,少なくともテーブルサイズ $10^0〜10^{8.3}$ において,
大きな違いがないため,uint16 の場合においても ${\rm limit}=254$ としており,$a$, $b$ にも同じ定数を用いる.


\section{ハッシュ値の計算}

ハッシュテーブルでは,キーからハッシュ値を生成し,テーブルサイズに収まるように丸める.
このとき,剰余演算を用いて丸めることが多い.
また,剰余演算を用いる場合には,ハッシュ値とテーブルサイズが互いに素となるよう素数にするのが望ましい \citep{石畑1989}.
しかしながら,
第\ref{chap_Introduction}章で示したように,
整数除算は探査速度が至上命題となる場合には,あまりにも遅い.
したがって,
dense\_hash\_map と同様に,
テーブルサイズを $2^k-1\ \ (k=1,2,...)$ とし,
ハッシュ値の最下位 $k$ ビットだけをビットマスクにより取り出して,
配列インデックスとする.


\begin{figure}[h]
  \includegraphics[scale=0.73]{./fig_algo/algorism_crop_06.pdf}
  \caption{ Insertion case01. }
  \label{fig_IpCHashT_insert_hard_case01}
%\end{figure}

%\begin{figure}[h]
  \includegraphics[scale=0.73]{./fig_algo/algorism_crop_07.pdf}
  \caption{ Insertion case02. }
  \label{fig_IpCHashT_insert_hard_case02}
%\end{figure}

%\begin{figure}[h]
  \includegraphics[scale=0.73]{./fig_algo/algorism_crop_08.pdf}
  \caption{ Insertion case03. }
  \label{fig_IpCHashT_insert_hard_case03}
%\end{figure}

%\begin{figure}[h]
  \includegraphics[scale=0.73]{./fig_algo/algorism_crop_09.pdf}
  \caption{ Insertion case04. }
  \label{fig_IpCHashT_insert_hard_case04}
\end{figure}

\begin{figure}[h]
  \includegraphics[scale=0.73]{./fig_algo/algorism_crop_10.pdf}
  \caption{ Insertion case05. }
  \label{fig_IpCHashT_insert_hard_case05}
%\end{figure}

%\begin{figure}[h]
  \includegraphics[scale=0.73]{./fig_algo/algorism_crop_11.pdf}
  \caption{ Insertion case06. }
  \label{fig_IpCHashT_insert_hard_case06}
%\end{figure}

%\begin{figure}[h]
  \includegraphics[scale=0.73]{./fig_algo/algorism_crop_12.pdf}
  \caption{ Insertion case07. }
  \label{fig_IpCHashT_insert_hard_case07}
%\end{figure}

%\begin{figure}[h]
  \includegraphics[scale=0.73]{./fig_algo/algorism_crop_13.pdf}
  \caption{ Insertion case08. }
  \label{fig_IpCHashT_insert_hard_case08}
\end{figure}

\begin{figure}[h]
  \includegraphics[scale=0.73]{./fig_algo/algorism_crop_14.pdf}
  \caption{ Insertion case09. }
  \label{fig_IpCHashT_insert_hard_case09}
%\end{figure}

%\begin{figure}[h]
  \includegraphics[scale=0.73]{./fig_algo/algorism_crop_15.pdf}
  \caption{ Insertion case10. }
  \label{fig_IpCHashT_insert_hard_case10}
%\end{figure}

%\begin{figure}[h]
  \includegraphics[scale=0.73]{./fig_algo/algorism_crop_16.pdf}
  \caption{ Insertion case11. }
  \label{fig_IpCHashT_insert_hard_case11}
\end{figure}

%---

\begin{figure}[h]
  \includegraphics[scale=0.73]{./fig_algo/algorism_crop_18.pdf}
  \caption{ Deletion case01. }
  \label{fig_IpCHashT_deletion_case01}
%\end{figure}

%\begin{figure}[h]
  \includegraphics[scale=0.73]{./fig_algo/algorism_crop_19.pdf}
  \caption{ Deletion case02. }
  \label{fig_IpCHashT_deletion_case02}
%\end{figure}

%\begin{figure}[h]
  \includegraphics[scale=0.73]{./fig_algo/algorism_crop_20.pdf}
  \caption{ Deletion case03. }
  \label{fig_IpCHashT_deletion_case03}
%\end{figure}

%\begin{figure}[h]
  \includegraphics[scale=0.73]{./fig_algo/algorism_crop_21.pdf}
  \caption{ Deletion case04. }
  \label{fig_IpCHashT_deletion_case04}
\end{figure}

\begin{figure}[h]
  \includegraphics[scale=0.73]{./fig_algo/algorism_crop_22.pdf}
  \caption{
    Deletion case05.
    The process of reducing doubly linked list fragmentation.
    Link distance is reduced by moving a distant element to nearby empty element found by sequential search.
    As the element moves, a new empty element is created.
    To completely eliminate fragmentation,
    it must be performed recursively at least until the linked distance reaches a local minimum.
    Therefore, the execution cost is very high.
    \\
    {\bf 図\ref{fig_IpCHashT_deletion_case05}}
    Deletion case05.
    双方向リストの断片化を低減する処理.
    遠くにある要素を,線形探査により探した近くの空き要素へ移動させることでリンク距離を短くする.
    要素が移動すると新しい空き要素ができるため,
    断片化を完全に解消するには,少なくともリンク距離が局所的最小値となるまで再帰的に実行する必要がある.
    このため,実行コストは非常に高い.
  }
  \label{fig_IpCHashT_deletion_case05}
%\end{figure}

%\begin{figure}[h]
  \includegraphics[scale=0.73]{./fig_algo/algorism_crop_23.pdf}
  \caption{
    Deletion case06.
    The process of reducing doubly linked list fragmentation.
    Link distance is reduced by moving a distant element to nearby empty element found by sequential search.
    As the element moves, a new empty element is created.
    To completely eliminate fragmentation,
    it must be performed recursively at least until the linked distance reaches a local minimum.
    Therefore, the execution cost is very high.
    \\
    {\bf 図\ref{fig_IpCHashT_deletion_case06}}
    Deletion case06.
    双方向リストの断片化を低減する処理.
    遠くにある要素を,線形探査により探した近くの空き要素へ移動させることでリンク距離を短くする.
    要素が移動すると新しい空き要素ができるため,
    断片化を完全に解消するには,少なくともリンク距離が局所的最小値となるまで再帰的に実行する必要がある.
    このため,実行コストは非常に高い.
  }
  \label{fig_IpCHashT_deletion_case06}
\end{figure}

%---


\chapter{実装}
\label{chap_Implementation}

第\ref{chap_Results}章で使用するベンチマーク用コードについて説明する.

\section{ベンチマーク用コード}

{\bf ソースコード}
\samepage \\ \indent
本研究で使用したソースコードを下記に示す.({\bf \color{red}メモ:現状では Private Repository.})
\begin{center}
  \url{https://github.com/admiswalker/InPlaceChainedHashTable-IpCHashT-}
\end{center}

{\bf ファイル構成}
\samepage \\ \indent
上記に示すソースコードのファイル構成を表\ref{table_fileDesc}に示す.
\begin{table}[h]
  \begin{center}
%    \fontsize{9pt}{10pt}\selectfont
    \fontsize{7pt}{10pt}\selectfont
    \caption{File organization}
    \begin{tabular}{ccc} \hline
      File name                  & Description                                  & Origin \rule[0pt]{0pt}{8pt} \\ \hline
      flat\_hash\_map-master     & Extract file of "flat\_hash\_map-master.zip" & \\
      googletest-master          & Extract file of "googletest-master.zip"      & \\
      sparsehash-master          & Extract file of "sparsehash-master.zip"      & \\
      sstd                       & Extract file of "sstd.zip"                   & \\
      CHashT.hpp                 & Inplimentation of "sstd::CHashT"             & \\
      FNV\_Hash.cpp              & Light claculation weight hash function       & $^{a)}$\url{qiita.com/Ushio/items/a19083514d087a57fc72} \\
      FNV\_Hash.hpp              & Light claculation weight hash function       & $^{a)}$\url{qiita.com/Ushio/items/a19083514d087a57fc72} \\
      IpCHashT.hpp               & Inplimentation of "sstd::IpCHashT" (Proposing method) & \\
      Makefile                   & Makefile                                     & \\
      README.md                  & Read me file                                 & \\
      bench.hpp                  & Benchmark                                    & \\
      bench\_main.cpp            & Entry potion for "bench.hpp"                 & \\
      exe\_bm                    & Binary file for benchmark                    & \\
      exe\_t                     & Binary file for test of "test\_CHashT.hpp" and "test\_IpCHashT.hpp" & \\
      flat\_hash\_map-master.zip & Inplimentation of "ska::flat\_hash\_map"     & $^{a)}$\url{github.com/skarupke/flat_hash_map} \\
      googletest-master.zip      & Google's C++ test framework                  & $^{a)}$\url{github.com/google/googletest} \\
      plots.py                   & Plotting funcrions for benchmark             & \\
      sparsehash-master.zip      & Inplimentation of "google::dense\_hash\_map" & $^{a)}$\url{github.com/sparsehash/sparsehash} \\
      sstd.zip                   & Convenient functions set                     & $^{a)}$\url{github.com/admiswalker/SubStandardLibrary} \\
      test\_CHashT.hpp           & Test code for "CHashT.hpp"                   & \\
      test\_IpCHashT.hpp         & Test code for "IpCHashT.hpp"                 & \\
      test\_main.cpp             & Entry potion for "test\_CHashT.hpp" and "test\_IpCHashT.hpp" & \\
      typeDef.h                  & Type definitions for integer & \\ \hline
    \end{tabular}
    \label{table_fileDesc}\\
    $^{a)}$ The protocol is "https".
  \end{center}
\end{table}

%\leavevmode \newline
%\leavevmode \newline
%\leavevmode \newline

\newpage
{\bf 環境}
\samepage\newline\indent
環境は Ubuntu を想定しており,コンパイルには G++ が必要である.
また,グラフのプロットには Python インタプリタと matplotlib が必要となる.
\\

{\bf 実行手順}
\samepage\newline\indent
実行手順は,図\ref{fig_command}に示す通りである.
\vspace{-2mm}
\begin{figure}[h]
  \hspace{2mm}
  \includegraphics[scale=0.73]{./fig_odg/libre_crop_01.pdf}
  \caption{
    ベンチマークの実行手順.
    ソースコードを Git リポジトリからクローンし,
    ディレクトリを移動,コンパイル,テストコードの実行,ベンチマークの計算,計算結果の統計処理
    を行っている.
  }
  \label{fig_command}
\end{figure}

{\bf sstd::IpCHashT のオプション}
\samepage\newline\indent
sstd::IpCHashT は,多数のオプションを備えており,ベンチマークで使用するオプションについては,
表\ref{table_alias}に示す alias を IpCHashT.hpp に定義する.
なお,IpCHashT\_u16 については,最大 load factor を 50 \% にする意味がないため,alias は設けない.
他のオプションは,IpCHashT.hpp に定義されており,\#define マクロにより,soft insertion か hard insertion と,
素数サイズのテーブルを用いて剰余演算するか,サイズ $2^k-1\ \ (k=1,2,...)$ のテーブルを用いてビットマスクするかを選択できる.

\begin{table}[h]
  \begin{center}
    \fontsize{9pt}{10pt}\selectfont
    \caption{sstd::IpCHashT aliases.}
    \begin{tabular}{c|ccc} \hline
      %                         Options
      Alias          & T\_shift & Maximum load factor [\%] \rule[0pt]{0pt}{10pt} & Search option \\ \hline
      IpCHashT\_u8hS & uint8    &  50 (half)                                     &   Successful search major option \rule[0pt]{0pt}{10pt} \\
      IpCHashT\_u8fS & uint8    & 100 (full)                                     &   Successful search major option \rule[0pt]{0pt}{10pt} \\
      IpCHashT\_u16S & uint16   & 100 (full)                                     &   Successful search major option \rule[0pt]{0pt}{10pt} \\
      IpCHashT\_u8hU & uint8    &  50 (half)                                     & Unsuccessful search major option \rule[0pt]{0pt}{10pt} \\
      IpCHashT\_u8fU & uint8    & 100 (full)                                     & Unsuccessful search major option \rule[0pt]{0pt}{10pt} \\
      IpCHashT\_u16U & uint16   & 100 (full)                                     & Unsuccessful search major option \rule[0pt]{0pt}{10pt} \\ \hline
    \end{tabular}
    \label{table_alias}
  \end{center}
\end{table}
%\\

{\bf ベンチマークのオプション}
\samepage\newline\indent
ベンチマークのオプションは,bench.hpp に定義されている.
Successful search major option か Unsuccessful search major option かの選択は,コメントアウトにより手動で行う.

軽くテストするには,limitSize を $5\times 10^6$ として \$ make, \$ ./exe\_bench を実行すると,
./tmpBench ディレクトリが生成され,10 分程で ベンチマーク結果が保存される.
グラフのスケールが合っていない場合は,plots.py を調整する必要がある.

本投稿に示すベンチマークは,limitSize を $2\times10^8$,loopNum を $100$ として,\$ make, \$ ./exe\_bench を実行する.
各ベンチマークを 100 回ずつ,tmpBench へ CSV ファイルとして出力するには,数日掛かる.
次に \$ ./exe\_sProc を実行すると,CSV ファイルから中央値が計算され,グラフとして出力される.
中央値を用いるのは,分布の形状が不明なこと,平均処理ではグラフのエッジが潰れること,ベンチマーク中の外れ値を除去することが目的である.
loopNum は 25 程度でも綺麗な結果が得られるため,まずは $25$ で試すとよい.
なお,''Allocated memory size'' だけは,メモリ量の都合により ${\rm limitSize}=2\times10^7$ でベンチマークする.






\chapter{ベンチマーク}
\label{chap_Results}

\section{環境}

ベンチマークコードの実行環境を,
表\ref{table_env}に示す.

\begin{table}[hbtp]
  \label{table_env}
  \begin{center}
    \caption{実行環境}
    \begin{tabular}{cc} \hline
      Component & Type \rule[0pt]{0pt}{0pt} \\ \hline
      CPU & AMD Ryzen7 1700 (8Cores/16Threads) \rule[0pt]{0pt}{0pt} \\ 
      & Base Clock 3GHz / Max Boost Clock 3.7GHz \rule[0pt]{0pt}{0pt} \\
      & Total L1 Cache: 768KB / Total L2 Cache: 4MB / Total L3 Cache: 16MB \\
      Memory & DDR4-2666 32GB \rule[0pt]{0pt}{0pt} \\
      OS & Ubuntu 16.04 LTS \rule[0pt]{0pt}{0pt} \\
      Compiler & gcc version 5.4.0 20160609 (Ubuntu 5.4.0-6ubuntu1~16.04.11) \rule[0pt]{0pt}{0pt} \\ \hline
    \end{tabular}
  \end{center}
\end{table}

\section{結果}

Loadfactor

メモリ使用量

挿入

探査

削除














\chapter{考察}
\label{chap_Discussion}

考察について記述する.
\leavevmode \newline

%
{\bf Load factor}
\samepage\newline\indent
Load factor とテーブルサイズの関係,図\ref{fig_bench_LF}について考察する.
{\bf std::unordered\_map} はテーブルサイズ $10^2$ 未満で Load factor が低いものの,
区間 $10^2〜10^8$ において,高い Load factor を示している.
{\bf sstd::CHashT} は区間 $10^1〜10^8$ に渡り 72.5 \% 以上を維持している.
{\bf sstd::IpCHashT (as maxLF50)},{\bf google::dense\_hash\_map},{\bf ska::flat\_hash\_map} は,
Load factor が制限されており,50 \% に留まっている.
{\bf sstd::IpCHashT (as uint8 and maxLF100)} は,テーブルサイズ $10^3$ 未満では 90 \% 以上の Load factor を示す.
テーブルサイズが $10^3$ を超えるにしたがい,linked list の長さが uint8 の最大値 - 1 に制限されている影響により,
Load factor が単調に減少していく.
{\bf sstd::IpCHashT (as uint16 and maxLF100)} は区間 $10^1〜10^8$ に渡り 97.5 \% 以上を維持している.
なお,実際には 100 サンプルの中央値のため,
特に sstd::CHashT と sstd::IpCHashT (as uint8 and maxLF100) では,
特にテーブルサイズが $10^3$ 未満の場合において,ある程度揺らぎがある.
また,sstd::IpCHashT の実装はいずれもは,パディングされる配列長も Load factor の計算に加算される.
このため,{\bf sstd::IpCHashT (as maxLF50)} では端数が発生し,丁度に 50 \% とはならない.
\leavevmode \newline

%
{\bf メモリ使用量}
\samepage\newline\indent
メモリ使用量とテーブルサイズの関係,図\ref{fig_bench_memory}について考察する.
複数のピークはハッシュテーブルのリハッシュを示す.
ピークの持つ幅は測定間隔に等しく,実際には要素1つ分の幅しか持たない.
{\bf std::unordered\_map} は要素数にしたがって,ほぼ線形にメモリ使用量を増加させている.
これは,std::unordered\_map が要素ごとにメモリを確保することを示す.
また,リハッシュ時のピークも小さいことから,
アルゴリズムは ``Separate chaining with linked lists'' であると推察される.
{\bf sstd::IpCHashT (as uint16 and maxLF100)} は
広い区間において最も高いメモリ効率を示しており,
図\ref{fig_bench_LF}で示した load factor の高さを反映する結果となった.
一部,他のハッシュテーブルとは異なるタイミングでリハッシュが発生しており,
load factor が 100 \% まで達していないことを示している.
{\bf google::dense\_hash\_map} は
Load factor が 50 \% に削減されているにも関わらず,
std::unordered\_map 前後のメモリ使用量を示しており,
key の一部を空符号や削除符号として使用する実装の特性が現れている.
{\bf sstd::CHashT},{\bf sstd::IpCHashT (as maxLF50)},{\bf ska::flat\_hash\_map} は,
いずれもほぼ同じメモリ使用量を示している.
図\ref{fig_bench_LF}より,
{\bf sstd::CHashT} は区間 $10^1〜10^8$ に渡り 72.5 \% 以上を維持しているものの,
ポインタによる singly linked list の構築に 1 要素あたり 8 Byte\footnote{64 bits CPU のため.} 必要としており,
多くのメモリを消費する結果となった.
Load factor の高い区間において,
線形にメモリ使用量が伸びており,
衝突した要素分のメモリを動的に確保している.


\noindent
{\bf sstd::IpCHashT (as maxLF50)},{\bf ska::flat\_hash\_map} は
検索速度を得るため,単位要素あたり最も多くのメモリを消費しており,
std::unordered\_map と比較して 1.5 倍程度となる.
実利用に際しては,このメモリ使用量を許容できるかどうかが,
一つの課題となる.
{\bf sstd::IpCHashT (as uint8 maxLF100)} は
sstd::IpCHashT (as maxLF50) と sstd::IpCHashT (as uint16 maxLF100) の
およそ中間でリハッシュする挙動を示している.
\leavevmode \newline

%
{\bf 挿入}
\samepage\newline\indent
挿入速度とテーブルサイズの関係,図\ref{fig_bench_insert_preAlloc},\ref{fig_bench_insert_wRehash},\ref{fig_bench_insert}について考察する.

テーブルのメモリを事前に確保した場合の挿入速度の累積時間は,
図\ref{fig_bench_insert_preAlloc} の通りである.
ただし,図\ref{fig_bench_insert} のように,挿入速度はテーブルサイズに依存しており,
ここでは,$2.0\times10^8$ に初期化されたテーブルに対する速度を示す.
まず,{\bf sstd::IpCHashT (as maxLF50)} と {\bf ska::flat\_hash\_map} は load factor が 50 \% を超えた時点でリハッシュしている.
{\bf google::dense\_hash\_map} についても,同じタイミングで load factor が 50 \% を超えているはずであるが,
事前にテーブルサイズを指定した場合には,リハッシュしないように制御されている様子が伺える.
{\bf sstd::IpCHashT (as uint8 and maxLF100)} については,
uint8 型により構成された singly linked list の最大範囲を超えた時点でリハッシュが発生している.
{\bf std::unordered\_map} は他のハッシュテーブルより 2\textasciitilde 3.5 倍程度遅い結果となった.
std::unordered\_map 以外のハッシュテーブルについて,計算量の違いは軽微である.

テーブルサイズを 0 で初期化した場合の挿入速度の累積時間は,
図\ref{fig_bench_insert_wRehash} の通りである.
図\ref{fig_bench_insert_preAlloc} と比較して,
{\bf sstd::CHashT} と {\bf sstd::IpCHashT (as uint16)} の累積時間の増加が顕著である.
これは,sstd::CHashT については,リハッシュそのものに時間が掛かっているためであり,
sstd::IpCHashT (as uint16) については,load factor の増加に伴い,空き要素の線形探査に時間が掛かるためである.
なお,他の条件の sstd::IpCHashT は,比較的 load factor の低い領域を使用するため,影響は軽微である.
{\bf std::unordered\_map} は,挿入とリハッシュの両方に時間が掛かっていることが伺える.
{\bf sstd::IpCHashT (as maxLF50)},{\bf sstd::IpCHashT (as uint8 and maxLF100)},{\bf ska::flat\_hash\_map},{\bf google::dense\_hash\_map} に
ついては,計算量の違いは軽微である.

テーブルサイズと挿入速度の関係は,図\ref{fig_bench_insert} に示した通りである.
キャッシュの変わり目を除き,広い区間で {\bf ska::flat\_hash\_map} が高い性能を示している.
{\bf google::dense\_hash\_map} は,L2 キャッシュに乗る区間 $10^2〜 10^5$ では,ska::flat\_hash\_map に次ぐ速度を示すものの,
L3 キャッシュ外となる区間 $10^6〜 10^8$ では,sstd::IpCHashT (as maxLF50) と入れ替わる結果となった.
{\bf sstd::IpCHashT (as maxLF50)} は,その挿入アルゴリズムの複雑さの割に高速な挿入である.
これは,結局のところ,挿入操作が単なる線形探査であるためである.
しかしながら,{\bf sstd::IpCHashT (as uint8 and maxLF100)} と {\bf sstd::IpCHashT (as uint16)} の示す通り,
load factor の高い領域において,線形探査は大きく効率を落としている.
{\bf sstd::CHashT} は L2,L3 キャッシュの外へ格納が増えるにしたがい,大きく速度を落としている.
\leavevmode \newline

%
{\bf 探査}
\samepage\newline\indent
探査速度とテーブルサイズの関係,図
\ref{fig_bench_find_s_sm},\ref{fig_bench_find_us_sm},
\ref{fig_bench_find_s_um},\ref{fig_bench_find_us_um}について考察する.
%探査速度とテーブルサイズの関係を,図
%\ref{fig_bench_find_s_sm},\ref{fig_bench_find_us_sm},
%\ref{fig_bench_find_s_um},\ref{fig_bench_find_us_um}に示す.
%図\ref{fig_bench_find_s_sm},\ref{fig_bench_find_us_sm}は Successful lookup を優先した設定,
%図\ref{fig_bench_find_s_um},\ref{fig_bench_find_us_um}は Unsuccessful lookup を優先した設定である.
\leavevmode \newline

%
{\bf 削除}
\samepage\newline\indent
まず,Successful lookup を優先した設定と,
Unsuccessful lookup の違いについて考察する.
本ベンチマークでは存在する key-value ペアを削除しているため,
図\ref{fig_bench_erase_sm}に示す Successful lookup を優先する設定が,
図\ref{fig_bench_erase_um}に示す Unsuccessful lookup を優先する設定より高速に処理すると期待されたが,
明白な違いは認められなかった.
%削除速度とテーブルサイズの関係を,図
%\ref{fig_bench_erase_sm},
%\ref{fig_bench_erase_um}に示す.
%図\ref{fig_bench_erase_sm}は Successful lookup を優先した設定,
%図\ref{fig_bench_erase_um}は Unsuccessful lookup を優先した設定である.
\leavevmode \newline



\chapter{結論}
\label{chap_Conclusion}

%結論について記述する.

本投稿では,
doubly linked list 構造を内包する Closed hashing アルゴリズムとして,
In-place Chained Hash Table を提案した.
図\ref{fig_taocp_v3_fig44}に示すように,
Chaining 系のアルゴリズムは, Successful lookup に加え,
特に Unsuccessful lookup に対して高い性能を示すことが期待された.
\newline

まず,{\bf sstd::IpCHashT (as uint8 and maxLF50)} の Unsuccessful lookup major option について結論を述べる.

Successful lookup speed について,
図\ref{fig_bench_find_s_um}より,
テーブルサイズ $1.0\times10^2〜1.0\times10^7$ において,
L2 キャッシュを跨ぐ $1.0\times10^5$ 前後を除き,
殆どの区間で,1 番目ないし,2 番目の実行速度を示した.
ただし,$1.0\times10^7$ を超える非常に巨大なハッシュテーブについては,
google::dense\_hash\_map がよい性能を示した.

Unsuccessful lookup speed については,
図\ref{fig_bench_find_us_um}より,
テーブルサイズ $1.0\times10^2〜1.0\times10^7$ において,
L2 キャッシュを跨ぐ $1.0\times10^5$ 前後を除き,
殆どの区間で,1 番目ないし,2 番目の実行速度を示した.
ただし,$1.0\times10^7$ を超える非常に巨大なハッシュテーブについては,
メモリ効率の最も高い sstd::IpCHashT (as uint16 and maxLF100) が,
最もよい性能を示した.

挿入速度に関しては,
図\ref{fig_bench_insert}に示す通り,必ずしも最速ではないものの,
図\ref{fig_bench_insert_wRehash}より,通常の使用において,
累積の挿入速度や,リハッシュ時間は極端に遅い訳ではないことを確認した.

削除速度に関しては,
図\ref{fig_bench_erase_um} に示す通り,
2\textasciitilde 3 番目の速度を示した.
\newline

{\bf sstd::IpCHashT (as uint8 and maxLF50)} の Successful lookup major option について結論を述べる.

Successful lookup speed について,
図\ref{fig_bench_find_s_sm}より,
Unsuccessful lookup major option の場合と比較して,
テーブルサイズ $3.0\times10^5〜1.0\times10^7$ において,
性能の改善が見られる.
それ以外の区間においては,大きな性能改善は見られず,
$1.0\times10^7$ を超える非常に巨大なハッシュテーブについては,
同様に性能が悪化した.

Unsuccessful lookup speed については,
図\ref{fig_bench_find_us_sm}より,
Unsuccessful lookup major option の場合と比較して,
L2 キャッシュサイズ内の $1.0\times10^2〜1.0\times10^5$ において,
大きく性能を落とした.
また,$1.0\times10^5〜1.0\times10^7$ においては,
ska::flat\_hash\_map が最も高い性能を示した.
\newline

{\bf sstd::IpCHashT (as uint16 and maxLF100)} の Successful lookup major option について結論を述べる.

Successful lookup speed について,
図\ref{fig_bench_find_s_sm}より,
区間 $1.0\times10^5〜3.5\times10^7$ において,
google::dense\_hash\_map と同程度の性能を示した.
区間 $3.5\times10^7〜2.0\times10^8$ においては,
google::dense\_hash\_map には劣るものの,
2 番目の性能に収まった.

Unsuccessful lookup speed については,
図\ref{fig_bench_find_s_sm}より,
区間 $1.0\times10^5〜3.5\times10^7$ において,
google::dense\_hash\_map と同程度の性能を示した.
区間 $3.5\times10^7〜2.0\times10^8$ においては,
最も速い性能を示した.

これらは,
google::dense\_hash\_map の 75 \% のメモリ使用量であることを鑑みれば,
よい結果であるといえる.

ただし,
挿入速度に関しては,
図\ref{fig_bench_insert} が示す通り,
load factor が高い場合に,極端に速度が低下するため,
図\ref{fig_bench_insert_wRehash} のように,
リハッシュを伴う要素の挿入には,
google::dense\_hash\_map の 1.7 倍程度の時間が掛かる.
もちろん,図\ref{fig_bench_insert_preAlloc} のように,
事前にハッシュテーブルが初期化されており,
load factor の比較的小さな領域を使用する場合には,
この限りではない.

削除速度に関しては,
図\ref{fig_bench_erase_sm} に示す通り,
sstd::CHashT よりは速い,程度の速度を保っている.




\leavevmode \newline
\leavevmode \newline
\leavevmode \newline
\leavevmode \newline
\leavevmode \newline
\leavevmode \newline
\leavevmode \newline
\leavevmode \newline
TODO: このあたりの結果を,いい感じの表にまとめる.







\chapter*{Appendix/付録} % 章番号を出さない
\addcontentsline{toc}{chapter}{Appendix/付録} % 目次に載せる

Appendix/付録.

% 付録は chapter の 1 つとして作りますが、章番号は表示しません。
% また付録の 1 つずつはアルファベットで番号付けをするのが一般的です。
\setcounter{section}{0} % section の番号をゼロにリセットする
\renewcommand{\thesection}{\Alph{section}} % 数字ではなくアルファベットで数える
\setcounter{equation}{0} % 式番号を A.1 のようにする
\renewcommand{\theequation}{\Alph{section}.\arabic{equation}}
\setcounter{figure}{0} % 図番号
\renewcommand{\thefigure}{\Alph{section}.\arabic{figure}}
\setcounter{table}{0} % 表番号
\renewcommand{\thetable}{\Alph{section}.\arabic{table}}

\section{セクション1}
内容.

\section{セクション2}
図/表など.

\include{08_Acknowledgments}

\renewcommand{\bibname}{参考文献} % jecon.bst だと bib の doi を読み込めずエラーを吐くので,ここだけ,削除すること.
\bibliographystyle{jecon}
\nocite{*} % 文献の引用がない場合のエラーを抑制.
\bibliography{09_References}

\end{document}
