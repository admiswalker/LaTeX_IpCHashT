\chapter{実装}
\label{chap_Implementation}

実装した内容について説明する.
実装上の注意等あればそれも説明する.
ベンチマーク周りの実装も軽く説明する.
\leavevmode \newline

{\bf ソースコード}
\samepage \\ \indent
本研究で使用したソースコードを下記に示す.({\bf \color{red}メモ:現状では Private Repository.})
\begin{center}
  \url{https://github.com/admiswalker/InPlaceChainedHashTable-IpCHashT-}
\end{center}

{\bf ファイル構成}
\samepage \\ \indent
あああ


{\bf 実行手順}
\samepage \\ \indent
実行手順は,図\ref{fig_command}に示す通りである.
\vspace{-2mm}
\begin{figure}[h]
  \hspace{2mm}
  \includegraphics[scale=0.73]{./figs_algo/algorism_crop_25.pdf}
  \caption{
    ベンチマークの実行手順.
    ソースコードを Git リポジトリからクローンし,
    ディレクトリを移動,コンパイル,テストコードの実行,ベンチマークの計算
    を行っている.
  }
  \label{fig_command}
\end{figure}
%\texttt{
%  \colorbox{black}{
%    \hbox to49zw{
%      \color{white}
%      \begin{tabular}{l}
%        \$ git clone git@github.com:admiswalker/InPlaceChainedHashTable-IpCHashT- \\
%        \$ cd ./InPlaceChainedHashTable-IpCHashT-\\
%        \$ make\\
%      \end{tabular}
%      \hfil}
%  }
%}
%  % colorbox: http://osksn2.hep.sci.osaka-u.ac.jp/~naga/miscellaneous/tex/tex-tips7.html
%  % fonts:  http://ideas.paunix.org/latex/latex_6_fontstyle.htm

