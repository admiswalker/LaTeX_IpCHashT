\chapter{実装}
\label{chap_Implementation}

第\ref{chap_Results}章で使用するベンチマーク用コードについて説明する.

\section{ベンチマーク用コード}

{\bf ソースコード}
\samepage \\ \indent
本投稿で使用したソースコードを下記に示す.({\bf \color{red}メモ:現状では Private Repository.})
\begin{center}
  \url{https://github.com/admiswalker/InPlaceChainedHashTable-IpCHashT-}
\end{center}

{\bf ファイル構成}
\samepage \\ \indent
上記に示すソースコードのファイル構成を Table \ref{table_fileDesc} に示す.
\begin{table}[h]
  \begin{center}
%    \fontsize{9pt}{10pt}\selectfont
%    \fontsize{8pt}{10pt}\selectfont
    \fontsize{7pt}{10pt}\selectfont
    \caption{
      File organization.
    }
    \begin{tabular}{cc} \hline
      File or directory name     & Description                                                           \rule[0pt]{0pt}{8pt} \\ \hline
      bench                      & Results of the benchmarks on this post                                \\
      tmpBenck                   & Output directory of the benchmarks                                    \\
      tmpDir                     & Temporary directory for graph plotting                                \\
      CHashT.hpp                 & Inplimentation of "sstd::CHashT"                                      \\
      FNV\_Hash.cpp              & Hash function for only implimentation test $^{a)}$                     \\
      FNV\_Hash.hpp              & Hash function for only implimentation test $^{a)}$                     \\
      IpCHashT.hpp               & Inplimentation of "sstd::IpCHashT" (Proposing method)                 \\
      Makefile                   & Makefile                                                              \\
      README.md                  & Read me file                                                          \\
      SubStandardLibrary-SSTD--master.zip & Convenient functions set $^{b)}$                              \\
      bench.hpp                  & Benchmark                                                             \\
      exe\_bench                 & Binary file for benchmark                                             \\
      exe\_bench\_uM             & Binary file for benchmarking allocated memory size called from exe\_bench \\
      exe\_sProc                 & Binary file for statistical process of "main\_sProc.cpp"              \\
      exe\_test                  & Binary file for test of "test\_CHashT.hpp" and "test\_IpCHashT.hpp"   \\
      flat\_hash\_map-master.zip & Inplimentation of "ska::flat\_hash\_map" $^{c)}$                       \\
      googletest-master.zip      & Google's C++ test framework $^{d)}$                                    \\
      main\_bench.cpp            & Entry point for benchmark of "bench.hpp"                              \\
      main\_bench\_usedMemory.cpp & Entry point for benchmark of allocated memory size of "bench.hpp"    \\
      main\_sProc.cpp            & Entry point for the benchmarks statistical process.                   \\
      main\_test.cpp             & Entry point for test of "test\_CHashT.hpp" and "test\_IpCHashT.hpp"   \\
      plots.py                   & Plotting funcrions for benchmark                                      \\
      sparsehash-master.zip      & Inplimentation of "google::dense\_hash\_map" $^{e)}$                   \\
      test\_CHashT.hpp           & Test code for "CHashT.hpp"                                            \\
      test\_IpCHashT.hpp         & Test code for "IpCHashT.hpp"                                          \\
      typeDef.h                  & Type definitions for integer                                          \\ \hline
    \end{tabular}
    \label{table_fileDesc}\\
    Origins: $^{a)}$\url{https://qiita.com/Ushio/items/a19083514d087a57fc72}, 
    $^{b)}$\url{https://github.com/admiswalker/SubStandardLibrary}, 
    $^{c)}$\url{https://github.com/skarupke/flat_hash_map}, 
    $^{d)}$\url{https://github.com/google/googletest},
    $^{e)}$\url{https://github.com/sparsehash/sparsehash}\\
%    $^{a)}$ The protocol is "https".
  \end{center}
\end{table}

%\leavevmode \newline
%\leavevmode \newline
%\leavevmode \newline

\newpage
{\bf 環境/Software Environment}
\samepage\newline\indent
For benchmarking, g++, git, make, cmake, and python with matplotlib is required.
\samepage\newline\indent
環境は Ubuntu を想定しており,コンパイルには G++ が必要である.
本体のビルドには Make が,
Google Test のビルドに CMake が必要となる.
また,グラフのプロットには Python インタプリタと matplotlib が必要となる.
\\

{\bf 実行手順}
\samepage\newline\indent
実行手順は,Fig. \ref{fig_command}に示す通りである.
%\vspace{-2mm}
\begin{figure}[h]
%  \hspace{2mm}
  \includegraphics[scale=0.73]{./fig_odg/libre_crop_01.pdf}
  \caption{
    The benchmarking process. \ \ 
    1. Cloning benchmark files from Git. \ \ 
    2. Chane directory. \ \ 
    3. Compile. \ \ 
    4. Run test code. \ \ 
    5. Run benchmark. \ \ 
    6. Run statistical processes.
    \\
    {\bf 図\ref{fig_command}}
    ベンチマークの実行手順.
    ソースコードを Git リポジトリからクローンし,
    ディレクトリを移動,コンパイル,テストコードの実行,ベンチマークの実行,計算結果の統計処理
    を行っている.
  }
  \label{fig_command}
\end{figure}

{\bf sstd::IpCHashT のオプション}
\samepage\newline\indent
sstd::IpCHashT は,多数のオプションを備えており,ベンチマークで使用するオプションについては,
Table \ref{table_alias} に示す alias を IpCHashT.hpp に定義する.
なお,IpCHashT\_u16 については,最大 load factor を 50 \% にする意味がないため,alias は設けない.
他のオプションは,IpCHashT.hpp に定義されており,\#define マクロにより,soft insertion か hard insertion と,
素数サイズのテーブルを用いて剰余演算するか,サイズ $2^k-1\ \ (k=1,2,...)$ のテーブルを用いてビットマスクするかを選択できる.

\begin{table}[h]
  \begin{center}
    \fontsize{9pt}{10pt}\selectfont
    \caption{sstd::IpCHashT aliases.}
    \begin{tabular}{c|ccc} \hline
      %                         Options
      Alias           & T\_shift & Maximum load factor [\%] \rule[0pt]{0pt}{10pt} & Search option \\ \hline
      IpCHashT\_u8hS  & uint8    &  50 (half)                                     &   Successful search major option \rule[0pt]{0pt}{10pt} \\
      IpCHashT\_u8fS  & uint8    & 100 (full)                                     &   Successful search major option \rule[0pt]{0pt}{10pt} \\
      IpCHashT\_u16hS & uint16   &  50 (half)                                     &   Successful search major option \rule[0pt]{0pt}{10pt} \\
      IpCHashT\_u16fS & uint16   & 100 (full)                                     &   Successful search major option \rule[0pt]{0pt}{10pt} \\
      IpCHashT\_u8hU  & uint8    &  50 (half)                                     & Unsuccessful search major option \rule[0pt]{0pt}{10pt} \\
      IpCHashT\_u8fU  & uint8    & 100 (full)                                     & Unsuccessful search major option \rule[0pt]{0pt}{10pt} \\
      IpCHashT\_u16hU & uint16   &  50 (half)                                     & Unsuccessful search major option \rule[0pt]{0pt}{10pt} \\
      IpCHashT\_u16fU & uint16   & 100 (full)                                     & Unsuccessful search major option \rule[0pt]{0pt}{10pt} \\ \hline
    \end{tabular}
    \label{table_alias}
  \end{center}
\end{table}
%\\

{\bf ベンチマークのオプション}
\samepage\newline\indent
ベンチマークのオプションは,bench.hpp に定義されている.
Successful search major option か unsuccessful search major option かの選択は,コメントアウトにより手動で行う.

軽くテストする場合は,limitSize を $5\times 10^6$ として \$ make, \$ ./exe\_bench を実行すると,
./tmpBench ディレクトリが生成され,10 分程で ベンチマーク結果が保存される.
グラフのスケールが合っていない場合は,plots.py を調整する必要がある.

本投稿に示すベンチマークは,limitSize を $2\times10^8$,loopNum を $100$ として,\$ make, \$ ./exe\_bench を実行する.
各ベンチマークを 100 回ずつ,tmpBench へ CSV ファイルとして出力するには,数日掛かる.
次に \$ ./exe\_sProc を実行すると,CSV ファイルから中央値が計算され,グラフとして出力される.
中央値を用いるのは,分布の形状が不明なこと,平均処理ではグラフのエッジが潰れること,ベンチマーク中の外れ値を除去することが目的である.
loopNum は 25 程度でも綺麗な結果が得られるため,まずは $25$ で試すとよい.


