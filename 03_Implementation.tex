\chapter{実装}
\label{chap_Implementation}

実装した内容について説明する.
実装上の注意等あればそれも説明する.
ベンチマーク周りの実装も軽く説明する.
\leavevmode \newline

\section{ベンチマーク用コード}

{\bf ソースコード}
\samepage \\ \indent
本研究で使用したソースコードを下記に示す.({\bf \color{red}メモ:現状では Private Repository.})
\begin{center}
  \url{https://github.com/admiswalker/InPlaceChainedHashTable-IpCHashT-}
\end{center}

{\bf ファイル構成}
\samepage \\ \indent
上記に示すソースコードのファイル構成を表\ref{table_fileDesc}に示す.
\begin{table}[h]
  \begin{center}
    \fontsize{7pt}{10pt}\selectfont
    \caption{File organization}
    \begin{tabular}{ccc} \hline
      File name                  & Description                                  & Origin \rule[0pt]{0pt}{8pt} \\ \hline
      flat\_hash\_map-master     & Extract file of "flat\_hash\_map-master.zip" & \\
      googletest-master          & Extract file of "googletest-master.zip"      & \\
      sparsehash-master          & Extract file of "sparsehash-master.zip"      & \\
      sstd                       & Extract file of "sstd.zip"                   & \\
      CHashT.hpp                 & Inplimentation of "sstd::CHashT"             & \\
      FNV\_Hash.cpp              & Light claculation weight hash function       & $^{a)}$\url{qiita.com/Ushio/items/a19083514d087a57fc72} \\
      FNV\_Hash.hpp              & Light claculation weight hash function       & $^{a)}$\url{qiita.com/Ushio/items/a19083514d087a57fc72} \\
      IpCHashT.hpp               & Inplimentation of "sstd::IpCHashT" (Proposing method) & \\
      Makefile                   & Makefile                                     & \\
      README.md                  & Read me file                                 & \\
      bench.hpp                  & Benchmark                                    & \\
      bench\_main.cpp            & Entry potion for "bench.hpp"                 & \\
      exe\_bm                    & Binary file for benchmark                    & \\
      exe\_t                     & Binary file for test of "test\_CHashT.hpp" and "test\_IpCHashT.hpp" & \\
      flat\_hash\_map-master.zip & Inplimentation of "ska::flat\_hash\_map"     & $^{a)}$\url{github.com/skarupke/flat_hash_map} \\
      googletest-master.zip      & Google's C++ test framework                  & $^{a)}$\url{github.com/google/googletest} \\
      plots.py                   & Plotting funcrions for benchmark             & \\
      sparsehash-master.zip      & Inplimentation of "google::dense\_hash\_map" & $^{a)}$\url{github.com/sparsehash/sparsehash} \\
      sstd.zip                   & Convenient functions set                     & $^{a)}$\url{github.com/admiswalker/SubStandardLibrary} \\
      test\_CHashT.hpp           & Test code for "CHashT.hpp"                   & \\
      test\_IpCHashT.hpp         & Test code for "IpCHashT.hpp"                 & \\
      test\_main.cpp             & Entry potion for "test\_CHashT.hpp" and "test\_IpCHashT.hpp" & \\
      typeDef.h                  & Type definitions for integer & \\ \hline
    \end{tabular}
    \label{table_fileDesc}\\
    $^{a)}$ The protocol is "https".
  \end{center}
\end{table}
% std::unordered_map<uint64,uint64>
% sstd::CHashT<uint64,uint64>
% sstd::IpCHashT<uint64,uint64>
% google::dense_hash_map<uint64,uint64>
% ska::flat_hash_map<uint64,uint64,ska::power_of_two_std_hash<uint64>>

%\leavevmode \newline
%\leavevmode \newline
%\leavevmode \newline

\newpage
{\bf 実行手順}
\samepage\newline\indent
実行手順は,図\ref{fig_command}に示す通りである.
\vspace{-2mm}
\begin{figure}[h]
  \hspace{2mm}
  \includegraphics[scale=0.73]{./fig_odg/libre_crop_01.pdf}
  \caption{
    ベンチマークの実行手順.
    ソースコードを Git リポジトリからクローンし,
    ディレクトリを移動,コンパイル,テストコードの実行,ベンチマークの計算,計算結果の統計処理
    を行っている.
  }
  \label{fig_command}
\end{figure}
%\texttt{
%  \colorbox{black}{
%    \hbox to49zw{
%      \color{white}
%      \begin{tabular}{l}
%        \$ git clone git@github.com:admiswalker/InPlaceChainedHashTable-IpCHashT- \\
%        \$ cd ./InPlaceChainedHashTable-IpCHashT-\\
%        \$ make\\
%      \end{tabular}
%      \hfil}
%  }
%}
%  % colorbox: http://osksn2.hep.sci.osaka-u.ac.jp/~naga/miscellaneous/tex/tex-tips7.html
%  % fonts:  http://ideas.paunix.org/latex/latex_6_fontstyle.htm



\section{ベンチマークの仕様}

ベンチマークにおける,
レイテンシとスループットの測定条件について説明する.
\leavevmode \newline

ハッシュテーブルの特性と
各操作における性能の測定条件について説明する.
\leavevmode \newline

{\bf 挿入}
\samepage \\ \indent
レイテンシの測定条件について説明する.
\leavevmode \newline

{\bf 探査}
\samepage \\ \indent
レイテンシの測定条件について説明する.
\leavevmode \newline

{\bf 削除}
\samepage \\ \indent
レイテンシの測定条件について説明する.
\leavevmode \newline

{\bf Load factor}
\samepage \\ \indent
レイテンシの測定条件について説明する.
\leavevmode \newline

{\bf メモリ使用量}
\samepage \\ \indent
レイテンシの測定条件について説明する.
\leavevmode \newline





%
% 第 3 章では,ベンチマークの実装がどのようであるかを説明し,
% 第 4 章では,その測定結果を,
% 第 5 章では,測定結果の考察を示す..
%

















