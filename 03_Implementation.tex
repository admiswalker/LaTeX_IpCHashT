\chapter{実装}
\label{chap_Implementation}

実装した内容について説明する.
実装上の注意等あればそれも説明する.
ベンチマーク周りの実装も軽く説明する.
\leavevmode \newline

{\bf ソースコード}
\samepage \\ \indent
本研究で使用したソースコードを下記に示す.({\bf \color{red}メモ:現状では Private Repository.})
\begin{center}
  \url{https://github.com/admiswalker/InPlaceChainedHashTable-IpCHashT-}
\end{center}

{\bf ファイル構成}
\samepage \\ \indent
\begin{table}[hbtp]
  \begin{center}
    \fontsize{9pt}{10pt}\selectfont
    \caption{ファイル構成.}
    \begin{tabular}{ccccccc} \hline
      File or directory name & Description \rule[0pt]{0pt}{10pt} \\ \hline
      flat\_hash\_map        &  \\ 
      googletest-master      &  \\
      sparsehash-master      &  \\
      sstd                   &  \\
      CHashT.hpp             &  \\
      FNV\_Hash.cpp          &  \\
      FNV\_Hash.hpp          &  \\
      IpCHashT.hpp           &  \\
      Makefile               &  \\
      README.md              &  \\
      bench.hpp              &  \\
      bench\_main.cpp        &  \\
      exe\_bm                &  \\
      exe\_t                 &  \\
      googletest-master.zip  &  \\
      plots.py               &  \\
      sparsehash-master.zip  &  \\
      sstd.zip               &  \\
      test\_CHashT.hpp       &  \\
      test\_IpCHashT.hpp     &  \\
      test\_main.cpp         &  \\
      typeDef.h              &  \\ \hline
    \end{tabular}
    \label{table_fileDesc}
  \end{center}
\end{table}



{\bf 実行手順}
\samepage \\ \indent
実行手順は,図\ref{fig_command}に示す通りである.
\vspace{-2mm}
\begin{figure}[h]
  \hspace{2mm}
  \includegraphics[scale=0.73]{./figs_odg/libre_crop_01.pdf}
  \caption{
    ベンチマークの実行手順.
    ソースコードを Git リポジトリからクローンし,
    ディレクトリを移動,コンパイル,テストコードの実行,ベンチマークの計算
    を行っている.
  }
  \label{fig_command}
\end{figure}
%\texttt{
%  \colorbox{black}{
%    \hbox to49zw{
%      \color{white}
%      \begin{tabular}{l}
%        \$ git clone git@github.com:admiswalker/InPlaceChainedHashTable-IpCHashT- \\
%        \$ cd ./InPlaceChainedHashTable-IpCHashT-\\
%        \$ make\\
%      \end{tabular}
%      \hfil}
%  }
%}
%  % colorbox: http://osksn2.hep.sci.osaka-u.ac.jp/~naga/miscellaneous/tex/tex-tips7.html
%  % fonts:  http://ideas.paunix.org/latex/latex_6_fontstyle.htm

