\chapter{ベンチマーク}
\label{chap_Results}

\section{環境}

ベンチマークの実行環境を,
表\ref{table_env}に示す.

\begin{table}[hbtp]
  \label{table_env}
  \begin{center}
    \caption{実行環境}
    \begin{tabular}{cc} \hline
      Component & Type \rule[0pt]{0pt}{0pt} \\ \hline
      CPU & AMD Ryzen7 1700 (8Cores/16Threads) \rule[0pt]{0pt}{0pt} \\ 
      & Base Clock 3GHz / Max Boost Clock 3.7GHz \rule[0pt]{0pt}{0pt} \\
      & Total L1 Cache: 768KB / Total L2 Cache: 4MB / Total L3 Cache: 16MB \\
      Memory & DDR4-2666 32GB \rule[0pt]{0pt}{0pt} \\
      OS & Ubuntu 16.04 LTS \rule[0pt]{0pt}{0pt} \\
      Compiler & gcc version 5.4.0 20160609 (Ubuntu 5.4.0-6ubuntu1~16.04.11) \rule[0pt]{0pt}{0pt} \\ \hline
    \end{tabular}
  \end{center}
\end{table}


\section{結果}

{\bf Loadfactor}


{\bf メモリ使用量}

{\bf 挿入}

{\bf 探査}

{\bf 削除}













