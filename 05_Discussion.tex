\chapter{考察}
\label{chap_Discussion}

考察について記述する.
\leavevmode \newline

%
{\bf Loadfactor}
\samepage\newline\indent
Load factor とテーブルサイズの関係は,図\ref{fig_bench_LF}に示す通りである.
まず,{\bf std::unordered\_map} はテーブルサイズ $10^2$ 未満で Load factor が低いものの,
区間 $10^2\textasciitilde 10^8$ において,高い Load factor を示している.
{\bf sstd::CHashT} は区間 $10^1\textasciitilde 10^8$ に渡り 72.5 \% を維持している.
{\bf sstd::IpCHashT (as uint8 and maxLF100)} は,テーブルサイズ $10^3$ 未満では 90 \% 以上の Load factor を示す.
テーブルサイズが $10^3$ を超えるにしたがい,linked list の長さが uint8 の最大値 - 1 に制限されている影響により,
Load factor が単調に減少していく.
なお,実際には 100 サンプルの中央値のため,
特に sstd::CHashT と sstd::IpCHashT (as uint8 and maxLF100) では,
特にテーブルサイズが $10^3$ 未満の場合において,ある程度揺らぎがある.
{\bf sstd::IpCHashT (as maxLF50)},{\bf google::dense\_hash\_map},{\bf ska::flat\_hash\_map} は,
Load factor が制限されており,50 \% に留まっている.
なお,sstd::IpCHashT の実装はいずれもは,パディングされる配列長も Load factor の計算に加算されるため,
端数が発生し,丁度に 50 \% とはならない.
\leavevmode \newline

%
{\bf メモリ使用量}
\samepage\newline\indent
%メモリー使用量とテーブルサイズの関係を,図\ref{fig_bench_memory}に示す.
\leavevmode \newline

%
{\bf 挿入}
\samepage\newline\indent
%挿入速度とテーブルサイズの関係を,図
%\ref{fig_bench_insert_preAlloc_sm},\ref{fig_bench_insert_wRehash_sm}に示す.
%挿入処理において,Hard insertion では find() 関数を用いるが,Soft insertion では,find() 関数を用いない.
%本ベンチマークでは,Soft insertion を用いる.
%したがって,Successful lookup と Unuccessful lookup オプションによる違いはない.
\leavevmode \newline

%
{\bf 探査}
\samepage\newline\indent
%探査速度とテーブルサイズの関係を,図
%\ref{fig_bench_find_s_sm},\ref{fig_bench_find_us_sm},
%\ref{fig_bench_find_s_um},\ref{fig_bench_find_us_um}に示す.
%図\ref{fig_bench_find_s_sm},\ref{fig_bench_find_us_sm}は Successful lookup を優先した設定,
%図\ref{fig_bench_find_s_um},\ref{fig_bench_find_us_um}は Unsuccessful lookup を優先した設定である.
\leavevmode \newline

%
{\bf 削除}
\samepage\newline\indent
まず,Successful lookup を優先した設定と,
Unsuccessful lookup の違いについて考察する.
本ベンチマークでは存在する key-value ペアを削除しているため,
図\ref{fig_bench_erase_sm}に示す Successful lookup を優先する設定が,
図\ref{fig_bench_erase_um}に示す Unsuccessful lookup を優先する設定より高速に処理すると期待されたが,
明白な違いは認められなかった.
%削除速度とテーブルサイズの関係を,図
%\ref{fig_bench_erase_sm},
%\ref{fig_bench_erase_um}に示す.
%図\ref{fig_bench_erase_sm}は Successful lookup を優先した設定,
%図\ref{fig_bench_erase_um}は Unsuccessful lookup を優先した設定である.
\leavevmode \newline


