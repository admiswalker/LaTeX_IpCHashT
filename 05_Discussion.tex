\chapter{考察}
\label{chap_Discussion}

最大 load factor,メモリ使用量,挿入速度,探査速度,削除速度について考察する.
\leavevmode \newline

%
%\section{最大 load factor}
{\bf 最大 load factor}
\samepage\newline\indent
図\ref{fig_bench_LF}に示す,最大 load factor とテーブルサイズの関係について考察する.
{\bf std::unordered\_map} はテーブルサイズ $10^2$ 未満で load factor が低いものの,
区間 $10^2〜10^8$ において,高い load factor を示している.
{\bf sstd::CHashT} は区間 $10^1〜10^8$ に渡り 72.5 \% 以上を維持している.
{\bf sstd::IpCHashT (as maxLF50)},{\bf google::dense\_hash\_map},{\bf ska::flat\_hash\_map} は,
load factor が制限されており,50 \% に留まっている.
{\bf sstd::IpCHashT (as uint8 and maxLF100)} は,テーブルサイズ $10^3$ 未満では 90 \% 以上の load factor を示す.
テーブルサイズが $4\times10^2$ を超える辺りから,linked list の長さが ``${\rm uint8} の最大値 - 1=254$'' に制限されている影響により,
load factor が単調に減少していく.
{\bf sstd::IpCHashT (as uint16 and maxLF100)} は区間 $10^1〜10^8$ に渡り 97.5 \% 以上を維持している.
なお,実際には 100 サンプルの中央値のため,
特に sstd::CHashT と sstd::IpCHashT (as uint8 and maxLF100) では,
特にテーブルサイズが $10^3$ 未満の場合において,ある程度揺らぎがある.
なお,sstd::IpCHashT の実装はいずれもパディングされる配列長を load factor の計算に加算している.
このため,{\bf sstd::IpCHashT (as maxLF50)} では端数が発生し,丁度に 50 \% とはならない.
\leavevmode \newline

%
%\section{メモリ使用量}
{\bf メモリ使用量}
\samepage\newline\indent
図\ref{fig_bench_memory}に示す,メモリ使用量とテーブルサイズの関係について考察する.
複数のピークはリハッシュを示す.
ピークは幅を持っているが,これは測定間隔に等しく,実際には要素 1 つ分の幅しか持たない.
{\bf std::unordered\_map} は要素数にしたがって,ほぼ線形にメモリ使用量を増加させている.
これは,std::unordered\_map が要素ごとにメモリを確保することを示す.
また,リハッシュ時のピークも小さいことから,
アルゴリズムは ``Separate chaining with linked lists'' であると推察される.
{\bf sstd::IpCHashT (uint16, maxLF100)} は
殆どの区間で最も高いメモリ効率を示しており,
図\ref{fig_bench_LF}で示した最大 load factor の高さを反映する結果となった.
一部,他のハッシュテーブルとは異なるタイミングでリハッシュが発生しており,
load factor が 100 \% まで達していないことを示している.
{\bf google::dense\_hash\_map} は
load factor が 50 \% に制限されているにも関わらず,
std::unordered\_map 前後のメモリ使用量を示しており,
キーの一部を空符号や削除符号として使用することで,
メモリ効率を高める実装の特性が現れている.
{\bf sstd::CHashT},{\bf sstd::IpCHashT (uint8, maxLF50)},{\bf ska::flat\_hash\_map} は,
いずれもほぼ同じメモリ使用量を示している.
図\ref{fig_bench_LF}より,
{\bf sstd::CHashT} は区間 $10^1〜10^8$ に渡り 72.5 \% 以上の最大 load factor を維持しているものの,
ポインタによる singly linked list の構築には,1 要素あたり 8 Byte\footnote{64 bits CPU のため.} 必要としており,
多くのメモリを消費する結果となった.
Load factor の高い区間において,
メモリ使用量が増加しており,
ハッシュ先が衝突した分のメモリを動的に確保している.

\noindent
{\bf sstd::IpCHashT (uint8, maxLF50)},{\bf ska::flat\_hash\_map} は
探査速度を得るため,単位要素あたり最も多くのメモリを消費しており,
std::unordered\_map と比較して 1.5 倍程度となる.
実利用に際しては,このメモリ使用量を許容できるかどうかが,
一つの課題となる.
{\bf sstd::IpCHashT (uint8, maxLF100)} は
sstd::IpCHashT (as maxLF50) と sstd::IpCHashT (as uint16 maxLF100) の
およそ中間でリハッシュする挙動を示している.

{\bf google::dense\_hash\_map}と{\bf sstd::IpCHashT (uint8, maxLF50)},{\bf sstd::IpCHashT (uint16)}の
理論値なメモリ使用量と図\ref{fig_bench_memory}に示す実測値を比較する.
%ここで,要素数を elems とする.
%key, value に uint64 を用いるとき,
まず,{\bf google::dense\_hash\_map}は,
\begin{eqnarray*}
  {\rm memory\ size|_{type=dense\_hash\_map}} &=& ({\rm key\ size} + {\rm value\ size}) \times ({\rm elements} / {\rm max\ load\ factor}) {\rm\ [Byte]}\\
  &=& ({\rm uint64} + {\rm uint64}) \times ({\rm elements} / 0.5) {\rm\ [Byte]}\\
  &=& (8 + 8) \times ({\rm elements} / 0.5) {\rm\ [Byte]}\\
  &=& 32 \times {\rm elements} {\rm\ [Byte]}
\end{eqnarray*}
となる.同様に,{\bf sstd::IpCHashT (uint8, maxLF50)}は,
\begin{eqnarray*}
  {\rm memory\ size|_{type=IpCHashT (uint8, maxLF50)}} &=& ({\rm key\ size} + {\rm value\ size} + {\rm next\ probe\ size} + {\rm prev\ probe\ size})\\
                                              & & \times ({\rm elements} / {\rm max\ load\ factor}) {\rm\ [Byte]}\\
  &=& ({\rm uint64} + {\rm uint64} + {\rm uint8} + {\rm uint8}) \times ({\rm elements} / 0.5) {\rm\ [Byte]}\\
  &=& (8 + 8 + 1 + 1) \times ({\rm elements} / 0.5) {\rm\ [Byte]}\\
  &=& 36 \times {\rm elements} {\rm\ [Byte]}
\end{eqnarray*}
となる.{\bf sstd::IpCHashT (uint16, maxLF100)}は,
\begin{eqnarray*}
  {\rm memory\ size|_{type=IpCHashT (uint16, maxLF100)}} &=& ({\rm key\ size} + {\rm value\ size} + {\rm next\ probe\ size} + {\rm prev\ probe\ size})\\
                                               & & \times ({\rm elements} / {\rm max\ load\ factor}) {\rm\ [Byte]}\\
  &=& ({\rm uint64} + {\rm uint64} + {\rm uint16} + {\rm uint16}) \times ({\rm elements} / 1.0) {\rm\ [Byte]}\\
  &=& (8 + 8 + 2 + 2) \times {\rm elements} {\rm\ [Byte]}\\
  &=& 20 \times {\rm elements} {\rm\ [Byte]}
\end{eqnarray*}
となる.

{\bf google::dense\_hash\_map}と{\bf sstd::IpCHashT (uint8, maxLF50)}の理論比を考えると,
\begin{eqnarray*}
  \frac{32 \times {\rm elements} {\rm\ [Byte]}}{36 \times {\rm elements} {\rm\ [Byte]}}
  = \frac{8}{9}\approx& 0.89
\end{eqnarray*}
となる.実測値は,図\ref{fig_bench_memory}より,要素数 $1.5\times 10^8$ のとき,8 GB と 12 GB であるから,
\begin{eqnarray*}
  \frac{8 {\rm\ [GByte]}}{12 {\rm\ [GByte]}}
  = \frac{2}{3}\approx& 0.67
\end{eqnarray*}
となる.したがって,理論値と実測値は一致しない.
これは,配列のアラインメントの問題だと考えられ,
IpCHashT では,コンパイル時に 2 つの uint8 のアラインメントが uint64 境界に合致するように,パディングされていると考えられる.
したがって,アラインメントを考慮して理論値を再計算すると,
{\bf sstd::IpCHashT (uint8, maxLF50)}を,
\begin{eqnarray*}
  {\rm memory\ size|_{type=IpCHashT (uint8, maxLF50)}} &=& ({\rm key\ size} + {\rm value\ size} + {\rm next\ probe\ size} + {\rm prev\ probe\ size})\\
                                              & & \times ({\rm elements} / {\rm max\ load\ factor}) {\rm\ [Byte]}\\
  &=& ({\rm uint64} + {\rm uint64} + {\rm uint8} + {\rm uint8}) \times ({\rm elements} / 0.5) {\rm\ [Byte]}\\
  &=& ({\rm uint64} + {\rm uint64} + {\rm uint64}) \times ({\rm elements} / 0.5) {\rm\ [Byte]}\\
  &=& (8 + 8 + 8) \times ({\rm elements} / 0.5) {\rm\ [Byte]}\\
  &=& 24 \times ({\rm elements} / 0.5) {\rm\ [Byte]}\\
  &=& 48 \times {\rm elements} {\rm\ [Byte]}
\end{eqnarray*}
とすると,
\begin{eqnarray*}
  \frac{32 \times {\rm elements} {\rm\ [Byte]}}{48 \times {\rm elements} {\rm\ [Byte]}}
  = \frac{2}{3}\approx& 0.67
\end{eqnarray*}
となる.したがって,理論値と実測値は一致する.

{\bf google::dense\_hash\_map}と{\bf sstd::IpCHashT (uint16, maxLF100)}の理論比についても同等であり,
64 bits 境界を考えて,{\bf sstd::IpCHashT (uint16, maxLF100)} の理論値を
\begin{eqnarray*}
  {\rm memory\ size|_{type=IpCHashT (uint16, maxLF100)}} &=& ({\rm key\ size} + {\rm value\ size} + {\rm next\ probe\ size} + {\rm prev\ probe\ size})\\
                                              & & \times ({\rm elements} / {\rm max\ load\ factor}) {\rm\ [Byte]}\\
  &=& ({\rm uint64} + {\rm uint64} + {\rm uint16} + {\rm uint16}) \times ({\rm elements} / 1.0) {\rm\ [Byte]}\\
  &=& ({\rm uint64} + {\rm uint64} + {\rm uint64}) \times ({\rm elements} / 1.0) {\rm\ [Byte]}\\
  &=& (8 + 8 + 8) \times {\rm elements} {\rm\ [Byte]}\\
  &=& 24 \times {\rm elements} {\rm\ [Byte]}
\end{eqnarray*}
となり,
\begin{eqnarray*}
  \frac{24 \times {\rm elements} {\rm\ [Byte]}}{32 \times {\rm elements} {\rm\ [Byte]}}
  = \frac{3}{4}= 0.75
\end{eqnarray*}
となる.実測値は,図\ref{fig_bench_memory}より,要素数 $1.5\times 10^8$ のとき,6 GB と 8 GB であるから,
\begin{eqnarray*}
  \frac{6 {\rm\ [GByte]}}{8 {\rm\ [GByte]}}
  = \frac{3}{4}= 0.75
\end{eqnarray*}
となる.したがって,理論値と実測値は一致する.


図\ref{fig_bench_memory_preAlloc}に示す,
ハッシュテーブルのサイズを事前指定して初期化した場合の,
メモリ使用量とテーブルサイズの関係について考察する.
この例では,
{\bf sstd::CHashT} が,テーブルサイズの増加に伴い,緩やかにメモリ使用量を増加させていることが分かる.
図\ref{fig_bench_memory} において,メモリ使用量の変化が折れ線のように見えるのは,
メモリ使用量を測定する際に解像度が不足していると考えられる.
また,
{\bf sstd::IpCHashT (as maxLF50)} と {\bf ska::flat\_hash\_map} について,
テーブルサイズは $2\times10^7$ より大きい最小の $2^k-1 (1,2,...)$ であるから,
$2^{26}-1 = 33554431$ とわかる.
これは,load factor が 50 \% のとき,
$33554431/2=16777215.5$ であり,
リハッシュが区間 $1.625\times10^7〜1.75\times10^7$ で発生していることにより裏付けられる.
一方で,{\bf google::dense\_hash\_map} は load factor を50 \% に制御するが,
メモリ使用量は,
テーブルサイズを指定して初期化した場合,倍の $2^{27}-1$ とわかる.

50 \% 以上まで要素を挿入していることが分かる.
\leavevmode \newline

%
%\section{挿入速度}
{\bf 挿入速度}
\samepage\newline\indent
図\ref{fig_bench_insert_preAlloc},\ref{fig_bench_insert_wRehash},\ref{fig_bench_insert}に示す,挿入速度とテーブルサイズの関係について考察する.

図\ref{fig_bench_insert_preAlloc} は,
テーブルのメモリを事前に確保した場合の挿入速度の累積時間である.
ただし,図\ref{fig_bench_insert} のように,挿入速度はテーブルサイズに依存しており,
ここでは,$2.0\times10^8$ に初期化されたテーブル
\footnote{
  正確には,$2.0\times10^8$ より大きな 2 の累乗の内,最も小さい値として,268435455 となる.
  実際にリハッシュしている区間は $1.250\times10^8〜1.375\times10^8$ であり,
  load factor が50 \% となる要素数 134217727 を含んでいる.
}
に対する速度を示す.
まず,{\bf sstd::IpCHashT (as maxLF50)} と {\bf ska::flat\_hash\_map} は load factor が 50 \% を超えた時点でリハッシュしている.
{\bf google::dense\_hash\_map} についても,同じタイミングで load factor が 50 \% を超えているはずであるが,
事前にテーブルサイズを指定した場合には,リハッシュしないように制御されている様子が伺える.
{\bf \color{red}メモ:この項目,以前削除してしまった,ベンチマークの図を挿入して,メモリ使用量が変化していないことから densh\_hash\_map がいろいろ制御されていることを示す.}
{\bf sstd::IpCHashT (as uint8 and maxLF100)} については,
uint8 型により構成された doubly linked list の最大範囲を超えた時点でリハッシュが発生している.
{\bf std::unordered\_map} は他のハッシュテーブルより 2\textasciitilde 3.5 倍程度遅い結果となった.
std::unordered\_map 以外のハッシュテーブルについて,計算量の違いは軽微である.

図\ref{fig_bench_insert_wRehash} は,
テーブルサイズを 0 で初期化した場合の挿入速度の累積時間である.
図\ref{fig_bench_insert_preAlloc} と比較して,
{\bf sstd::CHashT} と {\bf sstd::IpCHashT (as uint16)} の累積時間の増加が顕著である.
これは,sstd::CHashT については,リハッシュそのものに時間が掛かっているためであり,
sstd::IpCHashT (as uint16) については,load factor の増加に伴い,空き要素の線形探査に時間が掛かるためである.
なお,他の条件の sstd::IpCHashT は,比較的 load factor の低い領域を使用するため,影響は軽微である.
{\bf std::unordered\_map} は,挿入とリハッシュの両方に時間が掛かっていることが伺える.
{\bf sstd::IpCHashT (as maxLF50)},{\bf sstd::IpCHashT (as uint8 and maxLF100)},{\bf ska::flat\_hash\_map},{\bf google::dense\_hash\_map} に
ついては,計算量の違いは軽微である.

図\ref{fig_bench_insert} は,
テーブルサイズと挿入速度の関係である.
キャッシュの変わり目を除き,広い区間で {\bf ska::flat\_hash\_map} が高い性能を示している.
{\bf google::dense\_hash\_map} は,L2 キャッシュに乗る区間 $10^2〜 10^5$ では,ska::flat\_hash\_map に次ぐ速度を示すものの,
L3 キャッシュ外となる区間 $10^6〜 10^8$ では,sstd::IpCHashT (as maxLF50) と入れ替わる結果となった.
{\bf sstd::IpCHashT (as maxLF50)} は,その挿入アルゴリズムの複雑さの割に高速な挿入である.
これは,結局のところ,挿入操作が単なる線形探査であるためである.
しかしながら,{\bf sstd::IpCHashT (as uint8 and maxLF100)} と {\bf sstd::IpCHashT (as uint16)} の示す通り,
load factor の高い領域において,線形探査は大きく効率を落としている.
{\bf sstd::CHashT} は L2,L3 キャッシュの外へ格納が増えるにしたがい,大きく速度を落としている.
\leavevmode \newline

%
%\section{探査速度}
{\bf 探査速度}
\samepage\newline\indent
図
\ref{fig_bench_find_s_sm},\ref{fig_bench_find_us_sm},
\ref{fig_bench_find_s_um},\ref{fig_bench_find_us_um}に示す,
探査速度とテーブルサイズの関係について考察する.

%%%
まず,successful search の速度,図\ref{fig_bench_find_s_sm},\ref{fig_bench_find_us_sm}について考察する.

Successful major option を指定した速度は,図\ref{fig_bench_find_s_sm} である.
{\bf sstd::IpCHashT (as uint8 and maxLF50)} は,
L2 キャッシュから溢れるテーブルサイズ $10^5$ 前後を除き,
区間 $10^2〜10^7$ の殆どに渡り最速で探査している.
{\bf sstd::IpCHashT (as uint16)} は図\ref{fig_bench_memory} に示すように
google::dense\_hash\_map の 75 \% のメモリ使用量であるにも関わらず,
L3 キャッシュ内から CPU キャッシュ外の区間 $1.5\times10^5〜3.5\times10^7$ において,
同程度の速度を保持している.
ただし,区間 $1.0\times10^1〜1.5\times10^5$ と $3.5\times10^7〜2.0\times10^8$ においては,
google::dense\_hash\_map の方が高速に動作している.

Unsuccessful major option を指定した速度は,図\ref{fig_bench_find_us_sm} である.
{\bf sstd::IpCHashT (as uint8 and maxLF50)} は,
L2 キャッシュから溢れるテーブルサイズ $10^5$ 前後を除き,
区間 $10^2〜10^7$ の殆どに渡り 1\textasciitilde 2 番目の探査速度を保持している.
一方で,その他のオプションの sstd::IpCHashT は,優位性のある速度に達していない.

%%%
次に,unuccessful search の速度,図\ref{fig_bench_find_s_um},\ref{fig_bench_find_us_um}について考察する.

Successful major option を指定した速度は,図\ref{fig_bench_find_s_um} である.
{\bf sstd::IpCHashT (as uint8 and maxLF50)} は,
区間 $1.0\times10^2〜1.5\times10^5$ において,高速に動作するが,
区間 $1.5\times10^5〜1.5\times10^7$ においては,{\bf ska::flat\_hash\_map} が長い区間において,最速を保持している.
区間 $1.5\times10^7〜2.0\times10^8$ においては,よりメモリ効率の高い
{\bf sstd::IpCHashT (as uint8 and maxLF100)} や {\bf sstd::IpCHashT (as uint16)} が最速を保持している.
なお,図\ref{fig_taocp_v3_fig44} (a) に示すように,Separate Chaining は,本質的に unsuccessful search に強く,
一部区間では,{\bf sstd::CHashT} であっても,google::dense\_hash\_map を上回る速度を示している.

Unsuccessful major option を指定した速度は,図\ref{fig_bench_find_us_um} である.
{\bf sstd::IpCHashT (as uint8 and maxLF50)} は,
特に L1, L2 キャッシュに収まる区間 $1.0\times10^2〜1.5\times10^5$ では,
google::dense\_hash\_map の \textasciitilde4 倍程度高速に動作している.
また,
区間 $1.5\times10^5〜1.5\times10^7$ でも \textasciitilde3 倍程度高速に動作している.
一方で,
区間 $1.5\times10^7〜2.0\times10^8$ においては,よりメモリ効率の高い {\bf sstd::IpCHashT (as uint16)} が速度を伸ばした.
図\ref{fig_bench_find_us_um}の結果は特に,Chaining 系のアルゴリズムが高速に動作する結果となった.
\leavevmode \newline

%
%\section{削除速度}
{\bf 削除速度}
\samepage\newline\indent
図\ref{fig_bench_erase_sm},\ref{fig_bench_erase_um}に示す,削除速度とテーブルサイズの関係について考察する.

まず,速度の違いについて,
successful major option を指定した場合と,
unsuccessful major option を指定した場合の違いについて考察する.
このベンチマークでは,存在する key-value ペアを削除しているため,
図\ref{fig_bench_erase_sm}に示す successful search を優先する設定が,
図\ref{fig_bench_erase_um}に示す unsuccessful search を優先する設定より高速に処理すると期待されたが,
明白な違いは認められなかった.
これは,単に計算量がオーダーで異なるため,
違いが埋もれていると考えられる.

{\bf google::dense\_hash\_map} は,
区間 $1.0\times10^1〜2.0\times10^8$ のほぼ全てにおいて最速を保持している.
{\bf sstd::IpCHashT (as uint8 and maxLF50)} は,
L2 キャッシュ内の区間 $2.0\times10^2〜1.5\times10^5$ において,
ska::flat\_hash\_map より高速に動作するものの,
区間 $1.5\times10^2〜2.0\times10^8$ においては,
{\bf ska::flat\_hash\_map} の方が高速に動作している.
%\leavevmode \newline


