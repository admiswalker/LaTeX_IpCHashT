\chapter{アルゴリズム}
\label{chap_Algorism}

従来の Closed hashing では,
衝突ないし probing によりハッシュ値が使用中の場合,
本来1回の探査でアクセスできるはずの key であっても,
probing により衝突を解決しなくてはならない.
そこで,要素挿入時に,要素位置を整理して挿入する手法を提案する.
これは.挿入時間を犠牲に lookup 速度を向上させることを意味する.



\begin{figure}[h] % 特に強い理由がない限り、[htbp]のような指定はしないでください。
\begin{lstlisting}[language=C++]
template <class T_key, class T_val, typename T_shift>
struct element{
private:
public:
	element(){
		T_shift maxShift = (T_shift) 0; maxShift =~ maxShift;
		prev = maxShift;
		next = (T_shift) 0;
	}
	~element(){}
	
	T_key key;
	T_val val;
	T_shift prev;
	T_shift next;
};
\end{lstlisting}
\caption{
  In-place hash table 要素の C++ 擬似構造体.T\_key は key の型,T\_val は val の型である.
  T\_shift は doubly-linked list に用いる型で,uint8 または uint16 を指定する.
  uint16 より大きな型を指定するメリットは殆どない.
  prev は 前の要素までの相対距離を,next は 次の要素までの相対距離を表す.
  ポインタにおけるアドレスとは異なり,区間 [0, max(T\_shift) - 1] の範囲でリンクを表現する.
  ただし,max(T\_shift) は T\_shift 型の取り得る最大値である.
  0 のときに自分自身を示し,max(T\_shift) - 1 がリンクできる最大距離である.区間外へのリンクはできない.
  max(T\_shift) は,予約されており,'prev==max(T\_shift)' のとき,要素が空であることを示す.
  また,'prev==0' であれば linked list の先頭であること,'next==0' であれば linked list の末尾であることが分かる.
}
\label{fig_IpCHashT_struct}
\end{figure}



\section{挿入}

\begin{center}
  \includegraphics[page=1, scale=0.75]{./figs/algorism_crop.pdf}
\end{center}
\leavevmode \newline
\leavevmode \newline

%\begin{center}
%\includegraphics[page=2]{./figs/algorism_crop.pdf}
%\end{center}

\section{探査}
\section{削除}










