\chapter{アルゴリズム}
\label{chap_Algorism}

従来の Closed hashing では,
衝突ないし probing によりハッシュ値が使用中の場合,
本来1回目の探査でアクセスできるはずの key であっても,
probing により衝突を解決しなくてはならない.
そこで,要素挿入時に,要素位置を整理して挿入する手法を提案する.
これは.挿入時間を犠牲に lookup 速度を向上させることを意味する.
一般に,要素を先頭から末尾までただ辿る場合は,singly linked list を用いる.
今回は,後から挿入位置を整理するため,doubly linked list を用いる.

図 \ref{fig_IpCHashT_struct} に In-place hash table (以下,IpCHashT) のデータ構造を示す.
各要素には,key-value の他に,

\begin{figure}[h] % 特に強い理由がない限り、[htbp]のような指定はしないでください。
\begin{lstlisting}[language=C++]
template <class T_key, class T_val, typename T_shift>
struct element{
	element(){
		T_shift maxShift = (T_shift) 0; maxShift =~ maxShift;
		prev = maxShift;
		next = (T_shift) 0;
	}
	~element(){}
	
	T_key key;
	T_val val;
	T_shift prev;
	T_shift next;
};
\end{lstlisting}
\caption{
  IpCHashT 要素の C++ 擬似構造体.T\_key は key の型,T\_val は val の型である.
  T\_shift は doubly-linked list に用いる型で,uint8 または uint16 を指定する.
  uint16 より大きな型を指定するメリットは殆どない.
  prev は 前の要素までの相対距離を,next は 次の要素までの相対距離を表す.
  ポインタにおけるアドレスとは異なり,区間 [0, max(T\_shift) - 1] の範囲でリンクを表現する.
  ただし,max(T\_shift) は T\_shift 型の取り得る最大値である.
  0 のときに自分自身を示し,max(T\_shift) - 1 がリンクできる最大距離である.区間外へのリンクはできない.
  max(T\_shift) は,予約されており,'prev==max(T\_shift)' のとき,要素が空であることを示す.
  また,'prev==0' であれば linked list の先頭であること,'next==0' であれば linked list の末尾であることが分かる.
}
\label{fig_IpCHashT_struct}
\end{figure}



%\leavevmode \newline
%\leavevmode \newline
%  \vspace{-1cm}
\section{挿入}

\begin{figure}[h]
  \vspace{-0.5cm}
  \includegraphics[scale=0.73]{./figs_algo/algorism_crop_01.pdf}
  \caption{
    Description of the symbols.
  }
  \label{fig_IpCHashT_fig_description}
  \vspace{-0.5cm}
\end{figure}

\begin{figure}[h]
  \vspace{-0.5cm}
  \includegraphics[scale=0.73]{./figs_algo/algorism_crop_02.pdf}
  \caption{
    Insertion case01.
  }
  \label{fig_IpCHashT_insert_hard_case01}
  \vspace{-0.5cm}
\end{figure}

\begin{figure}[h]
  \vspace{-0.5cm}
  \includegraphics[scale=0.73]{./figs_algo/algorism_crop_03.pdf}
  \caption{
    Insertion case02.
  }
  \label{fig_IpCHashT_insert_hard_case02}
  \vspace{-0.5cm}
\end{figure}

\begin{figure}[h]
  \vspace{-0.5cm}
  \includegraphics[scale=0.73]{./figs_algo/algorism_crop_04.pdf}
  \caption{
    Insertion case03.
  }
  \label{fig_IpCHashT_insert_hard_case03}
  \vspace{-0.5cm}
\end{figure}

\begin{figure}[h]
  \vspace{-0.5cm}
  \includegraphics[scale=0.73]{./figs_algo/algorism_crop_05.pdf}
  \caption{
    Insertion case04.
  }
  \label{fig_IpCHashT_insert_hard_case04}
  \vspace{-0.5cm}
\end{figure}

\begin{figure}[h]
  \vspace{-0.5cm}
  \includegraphics[scale=0.73]{./figs_algo/algorism_crop_06.pdf}
  \caption{
    Insertion case05.
  }
  \label{fig_IpCHashT_insert_hard_case06}
  \vspace{-0.5cm}
\end{figure}

\begin{figure}[h]
  \vspace{-0.5cm}
  \includegraphics[scale=0.73]{./figs_algo/algorism_crop_07.pdf}
  \caption{
    Insertion case06.
  }
  \label{fig_IpCHashT_insert_hard_case11}
  \vspace{-0.5cm}
\end{figure}

\begin{figure}[h]
  \vspace{-0.5cm}
  \includegraphics[scale=0.73]{./figs_algo/algorism_crop_08.pdf}
  \caption{
    Insertion case07.
  }
  \label{fig_IpCHashT_insert_hard_case05}
\end{figure}

\begin{figure}[h]
  \vspace{-0.5cm}
  \includegraphics[scale=0.73]{./figs_algo/algorism_crop_09.pdf}
  \caption{
    Insertion case08.
  }
  \label{fig_IpCHashT_insert_hard_case07}
  \vspace{-0.5cm}
\end{figure}

\begin{figure}[h]
  \vspace{-0.5cm}
  \includegraphics[scale=0.73]{./figs_algo/algorism_crop_10.pdf}
  \caption{
    Insertion case09.
  }
  \label{fig_IpCHashT_insert_hard_case09}
  \vspace{-0.5cm}
\end{figure}

\begin{figure}[h]
  \vspace{-0.5cm}
  \includegraphics[scale=0.73]{./figs_algo/algorism_crop_11.pdf}
  \caption{
    Insertion case10.
  }
  \label{fig_IpCHashT_insert_hard_case08}
  \vspace{-0.5cm}
\end{figure}

\begin{figure}[h]
  \vspace{-0.5cm}
  \includegraphics[scale=0.73]{./figs_algo/algorism_crop_12.pdf}
  \caption{
    Insertion case11.
  }
  \label{fig_IpCHashT_insert_hard_case10}
  \vspace{-0.5cm}
\end{figure}





\section{探査}


\section{削除}
\begin{figure}[h]
  \vspace{-0.5cm}
  \includegraphics[scale=0.73]{./figs_algo/algorism_crop_14.pdf}
  \caption{
    Deletion case01.
  }
  \label{fig_IpCHashT_deletion_case01}
  \vspace{-0.5cm}
\end{figure}

\begin{figure}[h]
  \vspace{-0.5cm}
  \includegraphics[scale=0.73]{./figs_algo/algorism_crop_15.pdf}
  \caption{
    Deletion case02.
  }
  \label{fig_IpCHashT_deletion_case02}
  \vspace{-0.5cm}
\end{figure}

\begin{figure}[h]
  \vspace{-0.5cm}
  \includegraphics[scale=0.73]{./figs_algo/algorism_crop_16.pdf}
  \caption{
    Deletion case03.
  }
  \label{fig_IpCHashT_deletion_case03}
  \vspace{-0.5cm}
\end{figure}

\begin{figure}[h]
  \vspace{-0.5cm}
  \includegraphics[scale=0.73]{./figs_algo/algorism_crop_17.pdf}
  \caption{
    Deletion case04.
  }
  \label{fig_IpCHashT_deletion_case04}
  \vspace{-0.5cm}
\end{figure}

\begin{figure}[h]
  \vspace{-0.5cm}
  \includegraphics[scale=0.73]{./figs_algo/algorism_crop_18.pdf}
  \caption{
    Deletion case05.
  }
  \label{fig_IpCHashT_deletion_case05}
  \vspace{-0.5cm}
\end{figure}

\begin{figure}[h]
  \vspace{-0.5cm}
  \includegraphics[scale=0.73]{./figs_algo/algorism_crop_19.pdf}
  \caption{
    Deletion case06.
  }
  \label{fig_IpCHashT_deletion_case05}
  \vspace{-0.5cm}
\end{figure}

\section{配列確保(パディングの話)}









