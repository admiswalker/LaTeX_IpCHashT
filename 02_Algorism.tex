\chapter{アルゴリズム}
\label{chap_Algorism}

従来の Closed hashing では,
衝突ないし probing によりハッシュ値が使用中の場合,
本来1回目の探査でアクセスできるはずの key であっても,
probing により衝突を解決しなくてはならない.
そこで,要素挿入時に,要素位置を整理して挿入する手法を提案する.
これは.挿入時間を犠牲に lookup 速度を向上させることを意味する.
一般に,要素を先頭から末尾までただ辿る場合は,singly linked list を用いる.
今回は,後から挿入位置を整理するため,doubly linked list を用いる.

図 \ref{fig_IpCHashT_struct} に In-place hash table (以下,IpCHashT) のデータ構造を示す.
各要素には key-value pair の他に,linked list のための prev 要素と next 要素を備える.
T\_shift には unsigned int 型を用い,相対位置により linked list を構成する.
これは,ポインタ接続におけるメモリ消費量が無視できないためである.
\footnote{例えば 64 bits CPU の場合,ポインタサイズは 64 bits であるから,T\_key と T\_val が uint64 型の場合,
テープルサイズの 50\% が doubly linked list に由来する.}.

\begin{figure} % 特に強い理由がない限り、[htbp]のような指定はしないでください。
\begin{lstlisting}[language=C++]
template <class T_key, class T_val, typename T_shift>
struct element{
	element(){
		T_shift maxShift = (T_shift) 0; maxShift =~ maxShift;
		prev = maxShift;
		next = (T_shift) 0;
	}
	~element(){}
	
	T_key key;
	T_val val;
	T_shift prev;
	T_shift next;
};
\end{lstlisting}
\caption{
  IpCHashT 要素の C++ 擬似構造体.T\_key は key の型,T\_val は val の型である.
  T\_shift は doubly linked list に用いる型で,uint8 または uint16 を指定する.
  uint16 より大きな型を指定するメリットは殆どない.
  prev は 前の要素までの相対距離を,next は 次の要素までの相対距離を表す.
  ポインタにおけるアドレスとは異なり,区間 [0, max(T\_shift) - 1] の範囲でリンクを表現する.
  ただし,max(T\_shift) は T\_shift 型の取り得る最大値である.
  0 のときに自分自身を示し,max(T\_shift) - 1 がリンクできる最大距離である.区間外へのリンクはできない.
  max(T\_shift) は,予約されており,'prev==max(T\_shift)' のとき,要素が空であることを示す.
  また,'prev==0' であれば linked list の先頭であること,'next==0' であれば linked list の末尾であることが分かる.
}
\label{fig_IpCHashT_struct}
\end{figure}

\begin{figure}
  \includegraphics[scale=0.73]{./figs_algo/algorism_crop_01.pdf}
  \caption{ Description of the symbols. }
  \label{fig_IpCHashT_fig_description}
\end{figure}

図\ref{fig_IpCHashT_fig_description}に,
%図\ref{fig_IpCHashT_apparence}〜\ref{fig_IpCHashT_insert_introspection},
%図\ref{fig_IpCHashT_insert_hard_case01}〜\ref{fig_IpCHashT_insert_hard_case11} および
%図\ref{fig_IpCHashT_deletion_case01}〜\ref{fig_IpCHashT_deletion_case06}で用いる記号を示す.
図\ref{fig_IpCHashT_apparence}〜\ref{fig_IpCHashT_deletion_case06}で用いる記号を示す.
各図は,青字を初期状態とし,赤字が挿入時の変更を,緑字は要素の移動に伴う変更を,それぞれ表す.

図\ref{fig_IpCHashT_apparence}〜\ref{fig_IpCHashT_deletion_case06}では,
丸印ひとつ一つが配列要素を,
双方向に張られた2つの矢印が doubly linked list を表す.
Open hashing を用いたハッシュテーブルの説明では,
連続したアドレス空間上の配列要素が,
2辺を共有した四角形の連なりにより表現されることが多い.
同様に,IpCHashT では,配列要素を図\ref{fig_IpCHashT_apparence}のように抽象化して表すが,
一見すると隣接した連続構造であっても,
実際には,図\ref{fig_IpCHashT_insert_introspection}のように,
各要素が断続的なアドレス空間上に存在することがある.

\begin{figure}
  \includegraphics[scale=0.73]{./figs_algo/algorism_crop_03.pdf}
  \caption{
    IpCHashT に挿入された要素の抽象表現.
    この場合,3 つの要素のハッシュ値は,いずれも first アドレスを示すため,
    linked list により衝突を解決している.
    各要素は prev locator と next locator の示す相対位置により接続されており,
    各要素間に別の要素がある可能性がある.
  }
  \label{fig_IpCHashT_apparence}
\end{figure}

\begin{figure}
  \includegraphics[scale=0.73]{./figs_algo/algorism_crop_04.pdf}
  \caption{
    図\ref{fig_IpCHashT_apparence}に示す抽象表現を連続アドレス空間上に写像した場合の一例.
    同じ抽象表現でも,挿入と削除の手順により格納状態は異なる.
    ここでは,薄い灰色に囲まれた四角形が配列を示す.
    青字の linked list は図\ref{fig_IpCHashT_apparence}で示された構造を示し,
    赤字の linked list は間に挿入されていた別の chain を示す.
    この例では,手前に空の要素があるにも関わらず,
    青字の linked list の末尾が不用意に遠い場所に格納されており,
    断片化していることが分かる.
  }
  \label{fig_IpCHashT_insert_introspection}
\end{figure}

%\leavevmode \newline
%\leavevmode \newline
%\vspace{-1cm}
\section{挿入}

図\ref{fig_IpCHashT_insert_hard_case01}〜\ref{fig_IpCHashT_insert_hard_case11}に IpCHashT における挿入方法を示す.


\begin{figure}[h]
  \includegraphics[scale=0.73]{./figs_algo/algorism_crop_06.pdf}
  \caption{ Insertion case01. }
  \label{fig_IpCHashT_insert_hard_case01}
\end{figure}

\begin{figure}[h]
  \includegraphics[scale=0.73]{./figs_algo/algorism_crop_07.pdf}
  \caption{ Insertion case02. }
  \label{fig_IpCHashT_insert_hard_case02}
\end{figure}

\begin{figure}[h]
  \includegraphics[scale=0.73]{./figs_algo/algorism_crop_08.pdf}
  \caption{ Insertion case03. }
  \label{fig_IpCHashT_insert_hard_case03}
\end{figure}

\begin{figure}[h]
  \includegraphics[scale=0.73]{./figs_algo/algorism_crop_09.pdf}
  \caption{ Insertion case04. }
  \label{fig_IpCHashT_insert_hard_case04}
\end{figure}

\begin{figure}[h]
  \includegraphics[scale=0.73]{./figs_algo/algorism_crop_10.pdf}
  \caption{ Insertion case05. }
  \label{fig_IpCHashT_insert_hard_case05}
\end{figure}

\begin{figure}[h]
  \includegraphics[scale=0.73]{./figs_algo/algorism_crop_11.pdf}
  \caption{ Insertion case06. }
  \label{fig_IpCHashT_insert_hard_case06}
\end{figure}

\begin{figure}[h]
  \includegraphics[scale=0.73]{./figs_algo/algorism_crop_12.pdf}
  \caption{ Insertion case07. }
  \label{fig_IpCHashT_insert_hard_case07}
\end{figure}

\begin{figure}[h]
  \includegraphics[scale=0.73]{./figs_algo/algorism_crop_13.pdf}
  \caption{ Insertion case08. }
  \label{fig_IpCHashT_insert_hard_case08}
\end{figure}

\begin{figure}[h]
  \includegraphics[scale=0.73]{./figs_algo/algorism_crop_14.pdf}
  \caption{ Insertion case09. }
  \label{fig_IpCHashT_insert_hard_case09}
\end{figure}

\begin{figure}[h]
  \includegraphics[scale=0.73]{./figs_algo/algorism_crop_15.pdf}
  \caption{ Insertion case10. }
  \label{fig_IpCHashT_insert_hard_case10}
\end{figure}

\begin{figure}[h]
  \includegraphics[scale=0.73]{./figs_algo/algorism_crop_16.pdf}
  \caption{ Insertion case11. }
  \label{fig_IpCHashT_insert_hard_case11}
\end{figure}





\section{探査}


\section{削除}
\begin{figure}[h]
  \includegraphics[scale=0.73]{./figs_algo/algorism_crop_18.pdf}
  \caption{ Deletion case01. }
  \label{fig_IpCHashT_deletion_case01}
\end{figure}

\begin{figure}[h]
  \includegraphics[scale=0.73]{./figs_algo/algorism_crop_19.pdf}
  \caption{ Deletion case02. }
  \label{fig_IpCHashT_deletion_case02}
\end{figure}

\begin{figure}[h]
  \includegraphics[scale=0.73]{./figs_algo/algorism_crop_20.pdf}
  \caption{ Deletion case03. }
  \label{fig_IpCHashT_deletion_case03}
\end{figure}

\begin{figure}[h]
  \includegraphics[scale=0.73]{./figs_algo/algorism_crop_21.pdf}
  \caption{ Deletion case04. }
  \label{fig_IpCHashT_deletion_case04}
\end{figure}

\begin{figure}[h]
  \includegraphics[scale=0.73]{./figs_algo/algorism_crop_22.pdf}
  \caption{ Deletion case05. }
  \label{fig_IpCHashT_deletion_case05}
\end{figure}

\begin{figure}[h]
  \includegraphics[scale=0.73]{./figs_algo/algorism_crop_23.pdf}
  \caption{ Deletion case06. }
  \label{fig_IpCHashT_deletion_case06}
\end{figure}

\section{断片化}

\section{配列確保(パディングの話)}









