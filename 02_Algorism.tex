\chapter{アルゴリズム}
\label{chap_Algorism}

Linear probing や quadratic probing など
従来の closed hashing では,
ハッシュ先が重複していない場合でも,
別のキーの退避先として配列要素が使用中の場合,
本来 1 回目の探査でアクセスできるキーであっても,
probing により衝突を解決しなければならない.
Robin Hood hashing は,1 回でアクセスできる位置に要素を移動できるものの,
ハッシュ先の重複\footnote{secondary clustering.} に対しては線形探査が必要となる.

本投稿では,
ハッシュ先が別のキーの退避先として利用されてしまう primary clustering に付随した問題に対して,
逆方向にリストを辿ることで,要素が使用済みの場合でも適切に移動し,
ハッシュ先が必ず 1 つ目の要素となるように調整する.
また,要素探査時は,
順方向にリストを辿ることで平均探査回数を削減する.
メモリ使用量を削減するため,リストは int 型により相対位置を記録し,
これら順方向と逆方向を合わせて双方向リスト構造とする.
以上のアルゴリズムを実現するハッシュテーブルを in-place chained hash table と呼ぶこととし,
以降,IpCHashT と略記する.

IpCHashT のデータ構造における提案は,本質的に \cite{ADMIS2017} の提案と同様である.
以下,
Fig. \ref{fig_IpCHashT_struct},
Fig. \ref{fig_IpCHashT_insert_hard_case01}〜\ref{fig_IpCHashT_insert_hard_case11},
Fig. \ref{fig_IpCHashT_deletion_case01}〜\ref{fig_IpCHashT_deletion_case06}は,
\cite{ADMIS2017} に由来する.
ただし,本投稿では,
挿入アルゴリズムに新しく soft insertion を採用している.
探査においても,
successful search を優先するオプションと,
unsuccessful search を優先するオプションについて新たに実装と評価を行う.
また,パディングサイズは従来固定であったが,
テーブルサイズによってパディングサイズを動的に変更することで,
テーブルサイズが小さい場合に最大 load factor が低下する問題に対処する.
要素の挿入においては,操作手順を再検討し,より簡潔な実装とする.

IpCHashT では,Fig. \ref{fig_IpCHashT_struct}に示すデータ構造を備える.
各要素には key-value ペアの他に,双方向リストのための prev 要素と next 要素を持つ.
T\_shift には uint8 または uint16 型を用い,相対位置による双方向リストを構成する.
これは,ポインタ接続におけるメモリ消費量が無視できないためである
\footnote{
  例えば 64 bits CPU の場合,ポインタサイズは 64 bits であるから,T\_key と T\_val が uint64 型の場合,
  テープルサイズの 50\% が双方向リストに由来する.
}.
%\leavevmode \newline

\begin{figure} % 特に強い理由がない限り,[htbp]のような指定はしないでください。
\begin{lstlisting}[language=C++]
template <class T_key, class T_val, typename T_shift>
struct element{
	element(){
		T_shift maxShift = (T_shift) 0; maxShift =~ maxShift;
		prev = maxShift;
		next = (T_shift) 0;
	}
	~element(){}
	
	T_key key;
	T_val val;
	T_shift prev;
	T_shift next;
};
\end{lstlisting}
\caption{
  Pseudo C++ data structure for IpCHashT element. ``T\_key'' is a key type and ``T\_val'' is val a value type.
  ``T\_shift'' is used for doubly linked lists and specifies uint8 or uint16.
  Specifying a type larger than uint16 has little merit.
  ``prev'' indicates the distance from the previous element and 
  ``next'' indicates the distance from next element.
  Unlike the address in the pointer, the link list is expressed in the interval $[0, {\rm max(T\_shift)}-1]$,
  then max(T\_shift) is the maximum possible value of T\_shift type.
  ``$0$'' indicates itself and ``${\rm max(T\_shift)}-1$'' is the maximum link distance.
  There is no method to link outside of a section.
  ``max(T\_shift)'' is reserved for prev, and ``prev==max(T\_shift)'' indicates that the element is empty.
  ``prev==0'' indicates that the element is the head of the linked list,
  and ``next==0'' indicates that the element is the tail of the linked list.
  \\
  %%%
  {\bf 図\ref{fig_IpCHashT_struct}}
  IpCHashT 要素の C++ 擬似構造体.T\_key はキーの型,T\_val は val の型である.
  T\_shift は双方向リストに用いる型で,uint8 または uint16 を指定する.
  uint16 より大きな型を指定するメリットは殆どない.
  prev は 前の要素までの相対距離を,next は 次の要素までの相対距離を表す.
  ポインタにおけるアドレスとは異なり,区間 [0, max(T\_shift) - 1] の範囲でリンクを表現する.
  ただし,max(T\_shift) は T\_shift 型の取り得る最大値である.
  0 のときに自分自身を示し,max(T\_shift) - 1 がリンクできる最大距離である.区間外へのリンクはできない.
  max(T\_shift) は,予約されており,'prev==max(T\_shift)' のとき,要素が空であることを示す.
  また,'prev==0' であればリストの先頭であること,'next==0' であればリストの末尾であることがわかる.
}
\label{fig_IpCHashT_struct}
\end{figure}

\begin{figure}
  \includegraphics[scale=0.73]{./fig_algo/algorism_crop_01.pdf}
  \caption{
    Symbols used in Fig. \ref{fig_IpCHashT_apparence}〜\ref{fig_IpCHashT_deletion_case06}.
    `` Ope. '' indecates the execution order, and `` Ope. 0 '' means the initial state.
    The down arrow is placed on the array element indicated by the hash destination.
    Each element is composed by a circle, a prev locator, and a next locator.
    However, locators are omitted when there are no connection.
    Circles written with dotted lines represent empty elements.
    Green boxes and arrows represent movement of elements.
    Crosses indicate that the linke is deleted.
    The color scheme is blue for the initial state, red for element insertion and deletion, and green for element movement.
    \\
    {\bf 図\ref{fig_IpCHashT_fig_description}}
    Fig. \ref{fig_IpCHashT_apparence}〜\ref{fig_IpCHashT_deletion_case06}に用いる記号.
    ``Ope.''は,実行順序を表し,``Ope. 0'' の場合は初期状態を意味する.
    下向き矢印は,ハッシュ先が示す配列要素の上に置かれる.
    各要素は,丸 1 つと prev locator 1 つ,next locator 1 つで構成される.
    ただし,接続が無い場合 locator は省略される.
    点線で書かれた丸は空の要素を表す.また,緑色の枠線と矢印は要素の移動を表す.
    バツ印はリストの削除表す.
    配色は,青色を初期状態,赤色を要素の挿入と削除,緑色を要素の移動,としている.
  }
  \label{fig_IpCHashT_fig_description}
%\end{figure}
%\begin{figure}
  \includegraphics[scale=0.73]{./fig_algo/algorism_crop_03.pdf}
  \caption{
    An abstract representation of the element chain inserted into IpCHashT.
    In this case, the hash destinations of the three elements indicate the address of the first element,
    and the conflict is resolved by doubly linked list.
    Each element is connected by prev and next locators that indicate relative position, 
    and there may be another element between each element.
    \\
    {\bf 図\ref{fig_IpCHashT_apparence}}
    IpCHashT に挿入された要素の抽象表現.
    この場合,3 つの要素のハッシュ先は,いずれも first 要素のアドレスを示すため,
    双方向リストにより衝突を解決している.
    各要素は prev locator と next locator の示す相対位置により接続されており,
    各要素間に別の要素がある可能性がある.
  }
  \label{fig_IpCHashT_apparence}
%\end{figure}
%\begin{figure}
  \includegraphics[scale=0.73]{./fig_algo/algorism_crop_04.pdf}
  \caption{
    An example of mapping the abstract representation shown in Fig. \ref{fig_IpCHashT_apparence} to a continuous address space.
    A gray frame represents the array.
    Even with the same abstract representation,
    the state of the continuous address space depends on the array state at the time of insertion.
    Also, deleting elements can cause memory fragmentation.
    The blue list is a map of the abstract representation shown in Fig. \ref{fig_IpCHashT_apparence},
    and the red list is another chain inserted between them.
    In this example,
    the blue list is fragmented and the tail element is inserted in a distant place
    even though there is an empty element in the foreground.
    This fragmentation occurs when three or more elements are inserted into the blue list
    and then the value stored in the left empty element is deleted.
    \\
    {\bf 図\ref{fig_IpCHashT_apparence}}
    Fig. \ref{fig_IpCHashT_apparence}に示す抽象表現を連続アドレス空間上に写像した一例.
    灰色の枠線は配列を表す.
    同じ抽象表現でも,連続アドレス空間の状態は,要素挿入時の配列状態により異なる.
    また,要素の削除はメモリを断片化させることがある.
    青色のリストはFig. \ref{fig_IpCHashT_apparence}に示す抽象表現の写像であり,
    赤色のリストは間に挿入された別の chain である.
    この例では,青色のリストが断片化されており,手前に空の要素があるにも関わらず,末尾要素が遠い場所に格納されている.
    この断片化は,3 つ以上の要素が青色のリストに挿入された後に,
    左側の空要素に格納されていた値が削除されたことにより発生する.
  }
  \label{fig_IpCHashT_insert_introspection}
\end{figure}

本章で示すFig. \ref{fig_IpCHashT_fig_description}〜\ref{fig_IpCHashT_deletion_case06}は,
青色を初期状態とし,赤色が挿入時の変更を,緑色は要素の移動に伴う変更を,それぞれ表す.
Fig. \ref{fig_IpCHashT_fig_description}では,
Fig. \ref{fig_IpCHashT_apparence}〜\ref{fig_IpCHashT_deletion_case06}に用いる記号を説明している.
Fig. \ref{fig_IpCHashT_apparence}〜\ref{fig_IpCHashT_deletion_case06}では,
丸印ひとつ 1 つが配列要素を,
双方向に張られた2つの矢印が双方向リストを表す.
Open hashing を用いたハッシュテーブルの説明では,
連続したアドレス空間上の配列要素が,
2辺を共有した四角形の連なりにより表現されることが多い.
同様に,本投稿では IpCHashT の配列要素を抽象化して表すが,
一見するとFig. \ref{fig_IpCHashT_apparence}のように隣接した連続構造であっても,
実際には,Fig. \ref{fig_IpCHashT_insert_introspection}のように各要素が断続的なアドレス空間上に存在することがある.

%\leavevmode \newline
%\vspace{-1cm}
\leavevmode \newline
\section{挿入}

高い探査性能を達成するためには,
命令数を削減するだけでなく,
キャッシュミスを抑える必要がある.
一般に,CPU は配列アクセスに対して,
参照の局所性を利用してキャッシングする.
そのため,
必要な要素を配列に隙間なく詰め込むことで空間的局所性を,
可能な限り連続した位置に配置することで逐次的局所性を,
それぞれ高め,キャッシュミスを削減する.

key-value ペアを 1 つ挿入するには,
次の 1) 〜 4) の場合を考える.
1) ハッシュ先の配列が空の場合は,Fig. \ref{fig_IpCHashT_insert_hard_case01}のように単に要素を詰める.
2) 既に要素が挿入されており chain の間に空きがある場合は,Fig. \ref{fig_IpCHashT_insert_hard_case02}のように間に挿入する.
3) chain の間に空きがない場合は,Fig. \ref{fig_IpCHashT_insert_hard_case03}のように末尾に空きを探し挿入する.
4) 挿入先の要素が異なるハッシュ値を持つ要素の退避先に使用されている場合は,
Fig. \ref{fig_IpCHashT_insert_hard_case04}〜\ref{fig_IpCHashT_insert_hard_case11}のように挿入する.
要素の退避先と locator の再接続先との兼ね合いのため,4) には多くの場合分けが必要となる.

chain 間に空きは,線形探査により調べる.
これには,対象の要素を全て調べ上げる必要があり,処理に時間が掛かる.
また,11 通りの場合分けを全て実装するには非常に手間も掛かる.
Fig. \ref{fig_IpCHashT_insert_hard_case01}〜\ref{fig_IpCHashT_insert_hard_case11}に示す場合分けは,
任意の箇所に空きがあることを考慮している.
しかし,要素削除なく挿入する場合には,メモリの断片化は発生しないため,
必要な場合分けは
Fig. \ref{fig_IpCHashT_insert_hard_case01},
\ref{fig_IpCHashT_insert_hard_case03},
\ref{fig_IpCHashT_insert_hard_case06},
\ref{fig_IpCHashT_insert_hard_case11}
の 5 通りである.
また,要素削除を伴う場合においても,挿入コストが高いため推奨しない.
これは,要素の削除と再挿入によるメモリ断片化への耐性を捨て,
実装コストの低減と,挿入の高速化を優先している.


\section{探査}

挿入済みの要素を探査するには,Fig. \ref{alg_find_sm},\ref{alg_find_usm}に示すアルゴリズムが考えられる.
Fig. \ref{alg_find_sm}は successful search を優先した設定であり,以降,successful search major option と呼ぶ.
Fig. \ref{alg_find_usm}は unsuccessful search を優先した設定であり,以降,unsuccessful search major option と呼ぶ.
\samepage\newline\indent
Fig. \ref{alg_find_usm}の unsuccessful search major option は,
キーの比較コストによって successful search major option の結果より悪化することが十分に考えられるが,
比較コストの低いキーの場合には,unsuccessful search 時の分岐が少なく,高い性能が期待される.
実際に,分岐予測の失敗におけるペナルティは,10\textasciitilde 20 clock \footnote
{
  Fig. \ref{fig_part101_infographics}の ``Wrong'' branch of ``if'' (branch misprediction) を参照.
}
程度であり,無視できない.

\begin{figure}%[h] % 特に強い理由がない限り,[htbp]のような指定はしないでください。
\begin{lstlisting}[language=C++]
template <class T_key, class T_val>
(bool, T_val) find(T_key key_in){
	uint64 hashVal = hashFunc( key_in );
	uint64 idx = hashVal & tableSize_minus1;
	
	if(! isHead( table[ idx ] ) ){ return ( false, none ); }
	for(;;){
		if( table[ idx ].key == key_in ){ return ( true, table[ idx ].val ); }
		if( table[ idx ].next == 0 ){ return ( false, none ); }
		idx += table[ idx ].next;
	}
}
\end{lstlisting}
\caption{
  Pseudo C++ code for the ``find()'' function executed while {\bf successful search major option} is specified.
  This function is tuned to speed up ``successful searches'' more than ``unsuccessful seatches''.
  The ``isHead()'' function checks the hash destination is a head element or not, and if it is the head, it follows the list.
  At this time, the ``isHead()'' function returns true
  when the ``prev'' element which is a member element of the structure shown in Fig. \ref{fig_IpCHashT_struct} is $0$.
  Then the ``find()'' function returns true and a corresponding value to the key when a ``key\_in'' is equal to the key on the array.
  In all other cases, the ``find()'' function returns false.
  Compared with the existing closed hashing method,
  the number of key comparisons is limited to the number of elements on the doubly linked list,
  therefore high-speed successful search is possible.
  As with the dense\_hash\_map,
  the check with the ``isHead()'' function becomes unnecessary
  when one key is registered as an empty mark and is guaranteed not to be used.
  This is because the initial value of an element outside the list is guaranteed not to match any key.
  For this reason, removing the ``isHead()'' function will speed up unsuccessful search too while the key comparison cost is small.
  However, this post does not deal with such special conditions.
  \\
  {\bf 図\ref{alg_find_sm}}
  Successful search major option を指定した場合に実行される find 関数の擬似 C++ コード.
  Successful search が高速に実行されるようにチューニングされている.
  isHead 関数によりハッシュ先の要素が双方向リストの先頭か否かを確認し,先頭の場合に双方向リストを辿る.
  このとき,isHead 関数は,Fig. \ref{fig_IpCHashT_struct} に示す element 構造体の prev 要素が 0 のときに true を返す.
  リストの先頭の場合,key\_in が配列上のキーと一致すれば,対応する値と true を返す.
  それ以外の場合は全て false を返す.
  既存の closed hashing 法と比較して,キーの比較回数が双方向リスト上の要素数に制限されるため,高速な successful search が可能となる.
  なお,dense\_hash\_map と同様に,あるキーを空符号として登録し,使用されないことが保証される場合は,isHead 関数による検査が不要となる.
  これは,リスト外の要素の初期値が,いずれのキーとも一致しないことが保証されるためである.
  このため,isHead 関数を削除すると,キーの比較コストが小さい場合に,unsuccessful search も高速になる.
  ただし,条件が特殊なため,この投稿では扱わない.
}
\label{alg_find_sm}
\end{figure}

\begin{figure}%[h] % 特に強い理由がない限り,[htbp]のような指定はしないでください。
\begin{lstlisting}[language=C++]
template <class T_key, class T_val>
(bool, T_val) find(T_key key_in){
	uint64 hashVal = hashFunc( key_in );
	uint64 idx = hashVal & tableSize_minus1;
	
	for(;;){
		if( table[ idx ].key == key_in ){
			if( isEmpty( table[ idx ] ) ){ return ( false, none ); }
			return ( true, table[ idx ].val );
		}
		if( table[ idx ].next == 0 ){ return ( false, none ); }
		idx += table[ idx ].next;
	}
}
\end{lstlisting}
\caption{
  Pseudo C++ code for the ``find()'' function executed while {\bf unsuccessful search major option} is specified.
  This function is tuned to speed up ``unsuccessful searches'' more than ``successful seatches''.
  However, since the number of key comparisons is greater than that of successful search major options,
  data structures with high key comparison costs are considered to perform worse than successful search major options.
  In a case of unsuccessful search, it is necessary to find out all the keys associated with hash destination do not match ``key\_in''.
  Prioritizing key comparison makes unsuccessful search faster than ensuring the element is head.
  However, comparing the keys even if the hash destination is not a head of the list,
  comparing keys will cause a malfunction when a initial value of key element is equal to key\_in.
  For this reason, ``isEmpty()'' needs to check that the element is not empty.
  \\
  {\bf 図\ref{alg_find_usm}}
  Unsuccessful search major option を指定した場合に実行される find 関数の擬似 C++ コード.
  Unsuccessful search が高速に実行されるようにチューニングされている.
  ただし,キーの比較回数は successful search major option よりも多くなるため,
  キーの比較コストが高いデータ構造では,successful search major option より性能が悪化すると考えられる.
  Unsuccessful search では,キーの一致の有無に関わらず,ハッシュ先に関係するキーを全て探査し,key\_in と一致しないことを確認する必要がある.
  そのため,isHead より先頭要素であるかを確かめるよりも,キーの比較を優先する方が,高速に動作する.
  ところが,リストの先頭でない場合もキーを比較するため,
  キーが要素の初期値と等しい場合に誤動作する.
  このため,isEmpty により要素が空でないことを確認する必要がある.
}
\label{alg_find_usm}
\end{figure}

%\begin{algorithm}
%  \caption{Calculate $y = x^n$}
%  \label{alg1}
%  \begin{algorithmic}
%    \REQUIRE $n \geq 0 \vee x \neq 0$
%    \ENSURE $y = x^n$
%    \STATE $y \Leftarrow 1$
%    \IF{$n < 0$}
%    \STATE $X \Leftarrow 1 / x$
%    \STATE $N \Leftarrow -n$
%    \ELSE
%    \STATE $X \Leftarrow x$
%    \STATE $N \Leftarrow n$
%    \ENDIF
%    \WHILE{$N \neq 0$}
%    \IF{$N$ is even}
%    \STATE $X \Leftarrow X \times X$
%    \STATE $N \Leftarrow N / 2$
%    \ELSE[$N$ is odd]
%    \STATE $y \Leftarrow y \times X$
%    \STATE $N \Leftarrow N - 1$
%    \ENDIF
%    \ENDWHILE
%  \end{algorithmic}
%\end{algorithm}


\section{削除}

通常,リストの要素を削除するには,要素を削除した上で,ポインタを繋ぎ変えればよい.
しかし,IpCHashT では,幾つかの場合を考慮する必要がある.

key-value ペアを 1 つ削除するには,
次の 1) 〜 4) と,メモリの断片化を防ぐ処理として 5) を考える.
1) 単一要素の場合は,Fig. \ref{fig_IpCHashT_deletion_case01}のように単に削除する.
2) 末尾要素を削除する場合は,Fig. \ref{fig_IpCHashT_deletion_case02}のように要素と locator を削除する.
3) 先頭の要素が削除された場合,探査不能としないため,先頭に別の要素をつなぎ替える必要がある.
Fig. \ref{fig_IpCHashT_deletion_case03}では,断片化を防ぐために,末尾のデータを先頭へ移動させている.
4) chain の中間要素を削除する場合は,Fig. \ref{fig_IpCHashT_deletion_case04}のように末尾要素を移動させる.
5) 別の削除処理によって,末尾の要素が移動すると,
Fig. \ref{fig_IpCHashT_deletion_case05},\ref{fig_IpCHashT_deletion_case06}のように,
chain の要素間に空きができ,断片化する場合がある.これを
Fig. \ref{fig_IpCHashT_deletion_case05},\ref{fig_IpCHashT_deletion_case06}に示す操作を繰り返すことにより修正する.
ただし,空き要素の探査は線形探査する必要がある.
また,要素を移動させる度に別の chain に空きができる可能性があるため,再帰的に行う必要がある.
この処理は,実行コストが高いため推奨しない.

以上を勘案して,1) 〜 4) の操作を実装する.5) は実装しない.

% \section{断片化}
\section{配列の末尾処理}

要素の衝突が発生した際,closed hashing では,
現在より後のアドレスに key-value ペアを格納することで衝突を解決する.
しかし,ハッシュ値が配列の末尾を示した場合は,
退避先の配列要素が存在しない.
この場合,配列の末尾に達した場合は,
1) 引き続いて先頭から辿るように処理する,
2) 予め末尾に余分な配列を確保する,
の二択である.
まず,1) は,配列の読み込みが不連続となりキャッシュミスを誘発する.このため採用できない.
次に,2) は,キャッシュミスを誘発する危険はないものの,
余分な配列をどの程度確保するか問題となる.

2) で最も簡単な実装は,パディングを固定長とすることである.
ただし,これには欠点があり,テーブルサイズが小さいとき,
パディングが不足すると衝突を解決できずにリハッシュが発生し,
load factor の上限は上がらない,
逆に,パディング過剰の場合,
不用意にリハッシュが抑制され,探査効率が落ちる.

テーブルサイズが小さい場合,
パディングサイズが大きいいと,
テーブルサイズよりもパディングサイズが支配的となる.
逆に,テーブルサイズが大きい場合,
パディングサイズが小さいと,
退避先の配列が不足し,
最大 load factor が悪化する.
したがって,パディングサイズは,
テーブルサイズに応じて調整が必要となる.

必要なパディングサイズは,
線形に増加すると推測されるため,
\[
  {\rm pSize} = \begin{cases}
    (1/a) \cdot {\rm tSize} & ({\rm tSize}<{\rm limit}) \\
    {\rm limit}    & ({\rm tSize}\geq{\rm limit})
  \end{cases}
\]
\[
  a: {\rm threshold}, \ \ 
  {\rm pSize}: {\rm padding \ size}, \ \ 
  {\rm tSize}: {\rm table \ size}, \ \ 
  {\rm limit}: {\rm limit \ of \ T\_shift \ size}
\]
とすればよい.
しかし,
テーブルサイズが小さい場合ハッシュ先は十分に分散しないため,
単純に原点を通した調整では load factor の上限値が安定しない.
したがって,先ほどの数式に,バイアス項を付与した
\[
  {\rm pSize} = \begin{cases}
    (1/a) \cdot {\rm tSize} + b & ({\rm tSize}<{\rm limit}) \\
    {\rm limit}    & ({\rm tSize}\geq{\rm limit})
  \end{cases}
\]
\[
  a: {\rm threshold}, \ \ 
  b: {\rm bias}, \ \ 
  {\rm pSize}: {\rm padding \ size}, \ \ 
  {\rm tSize}: {\rm table \ size}
  {\rm limit}: {\rm limit \ of \ T\_shift \ size}
\]
をパディングの調整に用いる.

このとき,定数 $a$, $b$ は,load factor を最大化する,最小のパディングサイズとなる値が望ましい.
この定数が不用意に大きいと,IpCHashT は T\_shift の限界まで要素を挿入しようとする.
これは,特に,テーブルサイズが小さい場合に顕著となる.
また,uint16 を用いる場合は,その最大 T\_shift サイズ - 1 の 65534 要素先まで接続できてしまうため,
影響が広い区間に渡って続くことになる.
なお,この場合 65535 は空フラグとして予約されている.

定数 $a$, $b$ の最適値はともかくとして,
実用に耐えうる定数を探すだけであれば,
実験的に求められる.
これには,Fig. \ref{fig_bench_LF}に示す IpCHashT<uint64,uint64> (uint8, maxLF100) の load factor が最大となる,
最小のパディングサイズを探せばよい.
例えば,$a=18$, \ $b=35$, \ ${\rm limit}=254$ である.
なお,254 以上のパディングは,少なくともテーブルサイズ $10^0〜10^{8.3}$ において,
大きな違いがないため,uint16 の場合においても ${\rm limit}=254$ としており,$a$, $b$ にも同じ定数を用いる.


\section{ハッシュ値の計算}

ハッシュテーブルでは,キーからハッシュ値を生成し,テーブルサイズに収まるように丸める.
このとき,剰余演算を用いて丸めることが多い.
また,剰余演算を用いる場合には,ハッシュ値とテーブルサイズが互いに素となるよう素数にするのが望ましい \citep{石畑1989}.
しかしながら,
第\ref{chap_Introduction}章で示したように,
整数除算は探査速度が至上命題となる場合には,あまりにも遅い.
したがって,
dense\_hash\_map と同様に,
テーブルサイズを $2^k-1\ \ (k=1,2,...)$ とし,
ハッシュ値の最下位 $k$ ビットだけをビットマスクにより取り出して,
配列インデックスとする.


\begin{figure}[h]
  \includegraphics[scale=0.73]{./fig_algo/algorism_crop_06.pdf}
  \caption{ Insertion case01. }
  \label{fig_IpCHashT_insert_hard_case01}
%\end{figure}

%\begin{figure}[h]
  \includegraphics[scale=0.73]{./fig_algo/algorism_crop_07.pdf}
  \caption{ Insertion case02. }
  \label{fig_IpCHashT_insert_hard_case02}
%\end{figure}

%\begin{figure}[h]
  \includegraphics[scale=0.73]{./fig_algo/algorism_crop_08.pdf}
  \caption{ Insertion case03. }
  \label{fig_IpCHashT_insert_hard_case03}
%\end{figure}

%\begin{figure}[h]
  \includegraphics[scale=0.73]{./fig_algo/algorism_crop_09.pdf}
  \caption{ Insertion case04. }
  \label{fig_IpCHashT_insert_hard_case04}
\end{figure}

\begin{figure}[h]
  \includegraphics[scale=0.73]{./fig_algo/algorism_crop_10.pdf}
  \caption{ Insertion case05. }
  \label{fig_IpCHashT_insert_hard_case05}
%\end{figure}

%\begin{figure}[h]
  \includegraphics[scale=0.73]{./fig_algo/algorism_crop_11.pdf}
  \caption{ Insertion case06. }
  \label{fig_IpCHashT_insert_hard_case06}
%\end{figure}

%\begin{figure}[h]
  \includegraphics[scale=0.73]{./fig_algo/algorism_crop_12.pdf}
  \caption{ Insertion case07. }
  \label{fig_IpCHashT_insert_hard_case07}
%\end{figure}

%\begin{figure}[h]
  \includegraphics[scale=0.73]{./fig_algo/algorism_crop_13.pdf}
  \caption{ Insertion case08. }
  \label{fig_IpCHashT_insert_hard_case08}
\end{figure}

\begin{figure}[h]
  \includegraphics[scale=0.73]{./fig_algo/algorism_crop_14.pdf}
  \caption{ Insertion case09. }
  \label{fig_IpCHashT_insert_hard_case09}
%\end{figure}

%\begin{figure}[h]
  \includegraphics[scale=0.73]{./fig_algo/algorism_crop_15.pdf}
  \caption{ Insertion case10. }
  \label{fig_IpCHashT_insert_hard_case10}
%\end{figure}

%\begin{figure}[h]
  \includegraphics[scale=0.73]{./fig_algo/algorism_crop_16.pdf}
  \caption{ Insertion case11. }
  \label{fig_IpCHashT_insert_hard_case11}
\end{figure}

%---

\begin{figure}[h]
  \includegraphics[scale=0.73]{./fig_algo/algorism_crop_18.pdf}
  \caption{ Deletion case01. }
  \label{fig_IpCHashT_deletion_case01}
%\end{figure}

%\begin{figure}[h]
  \includegraphics[scale=0.73]{./fig_algo/algorism_crop_19.pdf}
  \caption{ Deletion case02. }
  \label{fig_IpCHashT_deletion_case02}
%\end{figure}

%\begin{figure}[h]
  \includegraphics[scale=0.73]{./fig_algo/algorism_crop_20.pdf}
  \caption{ Deletion case03. }
  \label{fig_IpCHashT_deletion_case03}
%\end{figure}

%\begin{figure}[h]
  \includegraphics[scale=0.73]{./fig_algo/algorism_crop_21.pdf}
  \caption{ Deletion case04. }
  \label{fig_IpCHashT_deletion_case04}
\end{figure}

\begin{figure}[h]
  \includegraphics[scale=0.73]{./fig_algo/algorism_crop_22.pdf}
  \caption{
    Deletion case05.
    The process of reducing doubly linked list fragmentation.
    Link distance is reduced by moving a distant element to nearby empty element found by sequential search.
    As the element moves, a new empty element is created.
    To completely eliminate fragmentation,
    it must be performed recursively at least until the linked distance reaches a local minimum.
    Therefore, the execution cost is very high.
    \\
    {\bf 図\ref{fig_IpCHashT_deletion_case05}}
    Deletion case05.
    双方向リストの断片化を低減する処理.
    遠くにある要素を,線形探査により探した近くの空き要素へ移動させることでリンク距離を短くする.
    要素が移動すると新しい空き要素ができるため,
    断片化を完全に解消するには,少なくともリンク距離が局所的最小値となるまで再帰的に実行する必要がある.
    このため,実行コストは非常に高い.
  }
  \label{fig_IpCHashT_deletion_case05}
%\end{figure}

%\begin{figure}[h]
  \includegraphics[scale=0.73]{./fig_algo/algorism_crop_23.pdf}
  \caption{
    Deletion case06.
    The process of reducing doubly linked list fragmentation.
    Link distance is reduced by moving a distant element to nearby empty element found by sequential search.
    As the element moves, a new empty element is created.
    To completely eliminate fragmentation,
    it must be performed recursively at least until the linked distance reaches a local minimum.
    Therefore, the execution cost is very high.
    \\
    {\bf 図\ref{fig_IpCHashT_deletion_case06}}
    Deletion case06.
    双方向リストの断片化を低減する処理.
    遠くにある要素を,線形探査により探した近くの空き要素へ移動させることでリンク距離を短くする.
    要素が移動すると新しい空き要素ができるため,
    断片化を完全に解消するには,少なくともリンク距離が局所的最小値となるまで再帰的に実行する必要がある.
    このため,実行コストは非常に高い.
  }
  \label{fig_IpCHashT_deletion_case06}
\end{figure}

%---

