\thispagestyle{empty} % ページ番号を表示しない.

% レイアウトを他の章のはじめのページと揃える.
 \\
\\
\\
\\
\noindent{\Huge \sf 要旨}\\
\\
\\
\\

ハッシュテーブルは,
ハッシュ化された入力キーを配列サイズで除算した際の余り (modulo) を配列のインデックスとして値を格納することで,
定数時間で高速にキーに対応する値を取得できるアルゴリズムである.
ハッシュテーブルは,キーと対応した値を保持する表であるため,配列のことをテーブルと表現する.

一般に,
ハッシュ値の剰余は配列全体に均一に分布することが望ましいが,
実際には限られた配列長に丸めるため衝突する.
%衝突確率を下げるには,より広い値空間に移せばよく,この操作をリハッシュと呼ぶ
%\footnote{リハッシュを行う際は,リハッシュ時間を定数時間に収めるため,通常は倍サイズの配列長に遷移させる.
%ただし,ここでの定数時間とは,繰り返し遷移させた場合のコストを1要素ごとで分割したコストが,単に O(n) となることを意味する.}
%.一般に,Chain 法.
そのため,衝突発生の際は,より広いハッシュ空間に移せばよく,この操作をリハッシュと呼ぶ.
しかし,衝突が偶発的であり,単純にリハッシュを繰り返せば,不用意にメモリを消費することから,
ハッシュ値が衝突した際にも,配列の利用率を示す座席利用率 (load factor) を上げる試みが行われてきた.

ハッシュ値が衝突した際に,key-value ペアを配列に押し込む手法は,大きく2つに分類される.
1つ目は,Chain 法に代表される Closed Addressing (別名:Open Hash) である.


2つ目は, 法に代表される Closed Addressing (別名:Open Hash) である.




