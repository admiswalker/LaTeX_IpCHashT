\thispagestyle{empty} % ページ番号を表示しない.

% レイアウトを他の章のはじめのページと揃える.
 \\
\\
\\
\\
\noindent{\Huge \sf 要旨}\\
\\
\\
\\

ハッシュテーブルは,
ハッシュ化された入力キーを配列サイズで除算した際の余り (modulo) を配列のインデックスとして値を格納することで,
キーに対応する値を,定数時間で高速に取得するアルゴリズムである.
ハッシュテーブルは,キーと対応した値を保持する表であることから,配列のことをテーブルと表現する.
また,キーと値の対応を key-value ペアと表現する.

一般に,
ハッシュ値の剰余は配列全体に均一に分布することが望ましいが,
実際には配列の利用率を示す座席利用率 (load factor) の増加に伴い衝突する.
%衝突確率を下げるには,より広い値空間に移せばよく,この操作をリハッシュと呼ぶ
%\footnote{リハッシュを行う際は,リハッシュ時間を定数時間に収めるため,通常は倍サイズの配列長に遷移させる.
%ただし,ここでの定数時間とは,繰り返し遷移させた場合のコストを1要素ごとで分割したコストが,単に O(n) となることを意味する.}
%.一般に,Chain 法.
剰余の衝突は,key-value ペアをより広い配列へ移すことで解決できる.
この操作をリハッシュと呼ぶ.
しかし,メモリ効率上,実用的な load factor を達成するためには,
リハッシュ以外の方法で衝突を解決する必要がある.

剰余衝突の解決策は,主に2種類に大別される.
1つ目は,Chaining に代表される Closed addressing (別名:Open hashing) である.
Chaining は,.

2つ目は,Linear probing や Quadratic probing に代表される Open addressing (別名:Closed hashing) である.
Linear probing は,.
これに対して,Quadratic probing は,.




