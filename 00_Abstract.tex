\thispagestyle{empty} % ページ番号を表示しない.

% レイアウトを他の章のはじめのページと揃える.
 \\
\\
\\
\\
\noindent{\Huge \sf 要旨}\\
\\
\\
\\

ハッシュテーブルは,
ハッシュ化された入力キーを配列サイズで除算した際の余り (modulo) を配列のインデックスとして値を格納することで,
キーに対応する値を,定数時間で高速に取得するアルゴリズムである.
ハッシュテーブルは,キーと対応した値を保持する表であることから,配列のことをテーブルと表現する.
また,キーと値の対応を key-value ペアと表現する.

一般に,
ハッシュ値の剰余は配列全体に均一に分布することが望ましいが,
実際には配列の利用率を示す座席利用率 (load factor) の増加に伴い衝突する.
%衝突確率を下げるには,より広い値空間に移せばよく,この操作をリハッシュと呼ぶ
%\footnote{リハッシュを行う際は,リハッシュ時間を定数時間に収めるため,通常は倍サイズの配列長に遷移させる.
%ただし,ここでの定数時間とは,繰り返し遷移させた場合のコストを1要素ごとで分割したコストが,単に O(n) となることを意味する.}
%.一般に,Chain 法.
剰余の衝突は,key-value ペアをより広い配列へ移すことで解決できる.
この操作をリハッシュと呼ぶ.
しかし,メモリ効率上,実用的な load factor を達成するためには,
リハッシュ以外の方法で衝突を解決する必要がある.

剰余衝突の解決策は,主に2種類に大別される.
1つ目は,
Chaining に代表される Open hashing\footnote{別名:Closed addressing.} である.
衝突が発生した際,新しくメモリ領域を確保し,片方向リスト\footnote{別名:linked list または singly linked list,} により現在の要素に追加する.
この手法は,
要素の削除と追加を繰り返す場合でも安定して動作するが,要素をポインタにより接続するため,
アドレス空間が連続とならず CPU キャッシュが効き難い欠点がある.
2つ目は,
Linear probing や Quadratic probing に代表される Closed hashing\footnote{別名:Open addressing.} である.
衝突が発生した際,それぞれの probing 規則に従い,隣接する空き要素に値を格納する.
Linear probing は,隣接する $1, 2, 3, ... k$ 番目の要素を線形探査する.
% 隣接する要素に値が格納されている場合は,高速に取り出せる.しかし,
配列が隙間なく埋まった場合は,
境界が分からず探査時間は大きく増加する.これを Primary clustering と呼ぶ.
% \footnote{Primary clustering と呼ぶ}.
% \href{https://en.wikipedia.org/wiki/Primary_clustering}{Primary clustering - wikipedia - 2019.06.22}}
% https://books.google.co.jp/books?id=HJ9gds_zhVEC&pg=PA186&redir_esc=y#v=onepage&q&f=false, ISBN 9780763725624 - Google Books
Quadratic probing は,隣接する $1^2, 2^2, 3^2, ... k^2$ 番目の要素を探査する.
飛び値を取るため,比較的 Primary clustering し難い.
いずれの probing も,空の要素が探査終了条件 \footnote{最悪計算量を保証するため,ホップ数を制限することもある.} の一つであるため,
要素を削除すると「要素が削除された」のか「要素が存在しない」のか判断できない.
このため,要素削除時は,削除フラグを付与する.
これらの手法は,
要素の削除と追加を繰り返す場合において,性能低下とそれに伴うリハッシュを必要とするが,
要素を隣接する要素に格納するため,CPU キャッシュが効き易い利点がある.




・chaining ポインタ接続による,キャッシュミスが課題.
・probing
そもそも本来挿入すべき場所に挿入できない.






