\chapter{結論}
\label{chap_Conclusion}

%結論について記述する.

本投稿では,
doubly linked list 構造を内包する Closed hashing アルゴリズムとして,
In-place Chained Hash Table を提案した.
図\ref{fig_taocp_v3_fig44}に示すように,
Chaining 系のアルゴリズムは, Successful lookup に加え,
特に Unsuccessful lookup に対して高い性能を示すことが期待された.
\newline

まず,{\bf sstd::IpCHashT (as uint8 and maxLF50)} の Unsuccessful lookup major option について結論を述べる.

Successful lookup speed について,
図\ref{fig_bench_find_s_um}より,
テーブルサイズ $1.0\times10^2〜1.0\times10^7$ において,
L2 キャッシュを跨ぐ $1.0\times10^5$ 前後を除き,
殆どの区間で,1 番目ないし,2 番目の実行速度を示した.
ただし,$1.0\times10^7$ を超える非常に巨大なハッシュテーブについては,
google::dense\_hash\_map がよい性能を示した.

Unsuccessful lookup speed については,
図\ref{fig_bench_find_us_um}より,
テーブルサイズ $1.0\times10^2〜1.0\times10^7$ において,
L2 キャッシュを跨ぐ $1.0\times10^5$ 前後を除き,
殆どの区間で,1 番目ないし,2 番目の実行速度を示した.
ただし,$1.0\times10^7$ を超える非常に巨大なハッシュテーブについては,
メモリ効率の最も高い sstd::IpCHashT (as uint16 and maxLF100) が,
最もよい性能を示した.

挿入速度に関しては,
図\ref{fig_bench_insert}に示す通り,必ずしも最速ではないものの,
図\ref{fig_bench_insert_wRehash}より,通常の使用において,
累積の挿入速度や,リハッシュ時間は極端に遅い訳ではないことを確認した.

削除速度に関しては,
図\ref{fig_bench_erase_um} に示す通り,
2\textasciitilde 3 番目の速度を示した.
\newline

{\bf sstd::IpCHashT (as uint8 and maxLF50)} の Successful lookup major option について結論を述べる.

Successful lookup speed について,
図\ref{fig_bench_find_s_sm}より,
Unsuccessful lookup major option の場合と比較して,
テーブルサイズ $3.0\times10^5〜1.0\times10^7$ において,
性能の改善が見られる.
それ以外の区間においては,大きな性能改善は見られず,
$1.0\times10^7$ を超える非常に巨大なハッシュテーブについては,
同様に性能が悪化した.

Unsuccessful lookup speed については,
図\ref{fig_bench_find_us_sm}より,
Unsuccessful lookup major option の場合と比較して,
L2 キャッシュサイズ内の $1.0\times10^2〜1.0\times10^5$ において,
大きく性能を落とした.
また,$1.0\times10^5〜1.0\times10^7$ においては,
ska::flat\_hash\_map が最も高い性能を示した.
\newline

{\bf sstd::IpCHashT (as uint16 and maxLF100)} の Successful lookup major option について結論を述べる.

Successful lookup speed について,
図\ref{fig_bench_find_s_sm}より,
区間 $1.0\times10^5〜3.5\times10^7$ において,
google::dense\_hash\_map と同程度の性能を示した.
区間 $3.5\times10^7〜2.0\times10^8$ においては,
google::dense\_hash\_map には劣るものの,
2 番目の性能に収まった.

Unsuccessful lookup speed については,
図\ref{fig_bench_find_s_sm}より,
区間 $1.0\times10^5〜3.5\times10^7$ において,
google::dense\_hash\_map と同程度の性能を示した.
区間 $3.5\times10^7〜2.0\times10^8$ においては,
最も速い性能を示した.

これらは,
google::dense\_hash\_map の 75 \% のメモリ使用量であることを鑑みれば,
よい結果であるといえる.

ただし,
挿入速度に関しては,
図\ref{fig_bench_insert} が示す通り,
load factor が高い場合に,極端に速度が低下するため,
図\ref{fig_bench_insert_wRehash} のように,
リハッシュを伴う要素の挿入には,
google::dense\_hash\_map の 1.7 倍程度の時間が掛かる.
もちろん,図\ref{fig_bench_insert_preAlloc} のように,
事前にハッシュテーブルが初期化されており,
load factor の比較的小さな領域を使用する場合には,
この限りではない.








